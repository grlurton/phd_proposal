\section[Understanding Information]{Aim 3 - Understanding information - what a bottom up appraoch is}

\subsection{Objective}

As a field, Global Health has to navigate between different levels of relevance. A Global Level, that aims at understanding global trends and phenomenons, and a local level, in which policies are implemented. In this regard, one of the challenges of Global Health is to create and disseminate information that will be relevant at these different levels. The difficulty in doing this is not merely a technical one, but also a political and structural one. The field in which this information is produced is indeed pre-organized by historic and political structures, that are not adapted for this multi-level adaptation of knowledge. National Statistical systems, for example, are the products of the larger systems they are designed to serve. As such, Health Information Systems are the results of the Health Systems in which they are created, and are influenced by the administrative cultures and the political context they are supposed to inform \citep{bergeron_savoirs_2014}.

This relation between local and global levels of Global Health is nonetheless asymmetric. As a scientific field, Global Health is essentially a hierarchical organization, where political, academic and economic institutions situated in high resource countries world offer guidance and validation for the methods and results used to measure health over the world. As a result, the field has to find an equilibrium between a downwards  \textit{standardization} of information and an upward \textit{aggregation} of locally produced data.

The social background in which this equilibrium is sought is important and has its roots in the origins of health information system in developing countries. Whereas European statistical systems development was led by social activists who guided the design of the first modern welfare systems \citep{desrosieres_politique_1993,desrosieres_administrator_1997}, colonial statistical systems were geared towards efficient land administration and economic exploitation \citep{rambert_cartographie_1922,de_martonne_cartographie_1931} with little attention to local population. The development of these statistical systems can indeed be traced to the enforcement of a specific mode of administrative control by colonial powers in the XIXth century \citep{appadurai_number_1996,cordell_couting_2010,gervais_how_2010}.
Colonial statisticians, often weakly skilled or trained \citep{kateb_gestion_1998,cordell_couting_2010}, nonetheless set the nomenclatures and conventions around how land and populations would be described and analyzed \citep{rambert_cartographie_1922,gervais_how_2010}. Enumeration and its categories did not emerge to describe local complexities and specifities, but were on the contrary simplifications aimed at creating a uniform colonial subject, to which a standard colonial rule would apply \citep{said_orientalism_1979,appadurai_number_1996}.


This historical background can be prolonged after the decolonizations, and can present some continuities with the way public health statistics are produced in developing countries today. Understanding how this longstanding history affects the way people involved in Global Health Metrics think about the production of quantitative evidence is important for practitioners of Global Health Metrics to understand the limits of their work, and the improvements they can bring.
% In these settings, strong public health information systems are the result of local equilibria and of adaptations to local statistical cultures, started by individuals making ad-hoc use of different data sources and inventing and standardizing methods on the run. \citep{lecuyer_medecins_1987}.

The aim of this paper will be to provide a clear understanding on how the structure of the field of Global Health can have an influence on how health metrics are produced, and how they are being used, or not used. Providing this framework on the conditions for designing quantitative evidence in Global Health is essential to design useful methods and metrics that can help shape public health policies at local and global levels. I will then use this framework to show how it can help understand and bring into context the results of my applied work.

%The emergence of Global Health Metrics as a field is aimed at providing "valid, reliable and comparable measures of the health states of individuals and of the health status of populations" \citep{mathers_population_2003} This triad of characteristics may be hard to ensure in a single scope, especially when the locally relevant may not be globally comparable. Meanwhile, this paper will explore the hypothesis that the emphasis given to comparability in this triade is historically, intellectually and politically situated, and can be linked to a culture of top-down standardization of knowledge in the developing world.


\subsection{Method}

This paper will build on the work and results of the two previous aims as well as on other work I have done in the field, to understand the conditions in which a \textsl{bottom-up} approach to health metrics can be defined and put into practice. The methodological challenge of this to work is to enter the domain of critique from within my own field \citep{latour_why_2004}. Most of the critical work on the use of quantitative evidence in health policy indeed comes from researchers that are not practitioners of Global Health Metrics \citep{merry_measuring_2011} . In this regard, even the most nuanced critique suffers the temptation of an overwhelming challenge to the possibility of quantitative work\citep{latour_why_2004}.

I will thus offer a framework and examples for an internal critique of this work. Using the opposition between a downward, normative movement and an upward, aggregative movement of methods and data, I will offer a simple explanation of how power relations that structure Global Health Metrics can be found in the methods that are used, and I will show how understanding the limits of these methods can help challenge this power structure, to reequilibrate the relationship between local and global actors of Global Health.

This work will be built in three main parts:

\begin{enumerate}
\item Provide a description of how a top-down approach to Health Metrics differs from a bottom-up approach.
\item Delineate how the structures in place in Global Health can influence which of these approach will be used by Global Health actors.
\item Describe the comparative benefits and shortcomings of each approach, and how they can be combined.
\end{enumerate}

The main methodological challenge of this paper is to make a common reading of literature from \gls{sts}, Colonial Studies and Global Health Metrics. This diverse corpus will be used to test the hypothesis made in other projects in this dissertation, as well as other projects I am involved in. I organize this project in four steps:

\begin{enumerate}
\item Explore the literature in \gls{sts} and Colonial Studies pertinent to the opposition \textit{bottom-up} vs \textit{top-down} of statistical methods.
\item Identify the themes and concepts relevant to the Health Metrics field in this literature.
\item Identify features relevant to these themes in my applied work, and build a critical framework in which to read this my projects.
\end{enumerate}

\subsection{Timeline}
\label{timeline:aim3}

I have already made most of the literature review for this work. A first paper on this work has been accepted for presentation in a workshop at the Ecole des Hautes Etudes en Sciences Sociales. Another presentation of this work for social scientists interested in Data Science should be held at the UW in december. This will allow me to get feedback and to complete my analytical framework. Most of the remaining work for this aim is to include the results from the two other aims and to situate them in my analytic framework.


\begin{figure}[!t]
	\begin{ganttchart}[vgrid,hgrid,y unit chart=.6cm]{1}{9}
		\gantttitle{2017}{3}
		\gantttitle{2018}{6} \\
		\gantttitlelist{10,...,12}{1} \gantttitlelist{1,...,6}{1}\\

		\ganttset{bar/.append style={draw=blue!40 , fill=blue!40},
					group/.append style={draw=blue, fill=blue}}
		\ganttbar{Finalize Litterature review}{1}{4} \\
		\ganttbar{Finalize analytical Framework}{1}{6} \\
		\ganttmilestone{Seminar Presentation}{3} \\
		\ganttbar{Paper finalization}{4}{8} \\
	\end{ganttchart}
	\caption{Gantt Chart for Aim 3}
	\label{GanttPaper3}
\end{figure}
