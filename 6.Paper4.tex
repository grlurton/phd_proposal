\section{Data Use}

construction d'un espace politique d'équivalence et de codage


"Les outils statistiques permettent de découvrir ou de créer des êtres sur lesquels prendre appui pour décrire le monde et agir sur lui"

\subsection{Information in Health Systems}




\begin{figure}[ht]
\begin{minipage}{.4\textwidth}
\begin{tikzpicture}[node distance=.8cm,  start chain=going below,]
     \node[punktchain, join] (DataCollection) {Data Collection};
     \node[punktchain, join] (DataManagement) {Data Transmission and Management};
     \node[punktchain, join] (DataAnalysis)   {Data Analysis};
     \node[punktchain, join] (DataUse) 		  {Data Use};
     \filldraw[ultra thick, draw=black, fill=green, opacity=0.2] (-2.2,-6.7) -- (-2.2,-5.3) -- (2.2,-5.3) -- (2.2,-6.7) -- (-2.2,-6.7) ;
\end{tikzpicture}
\end{minipage}
\begin{minipage}{.5\textwidth}
\begin{tikzpicture}[node distance=2cm]
\coordinate (A) at (-4.5,0) {};
\coordinate (B) at ( 4.5,0) {};
\coordinate (C) at ( 0,7.7942) {};
\draw[name path=AC] (A) -- (C);
\draw[name path=BC] (B) -- (C);
\draw (1.1,5.8971)--(3.5,5.8971) ;
\draw(3.5,5.8971)--(3.5,3) ;
\draw [->](3.5,3)--(2.77,3) ;
\node at (4.4,4.25) {Feedback};
\foreach \y/\A/\txtHigh in {0/Patients Care/0.8 ,2/Facility Administration \\ and Reporting/2.5,4/Planning \\ Monitoring \\ \& Evaluation /4.8}{
    \path[name path=horiz] (A|-0,\y) -- (B|-0,\y);
    \draw[name intersections={of=AC and horiz,by=P},
          name intersections={of=BC and horiz,by=Q}] (P) -- (Q)
          node[align = center,above] at (0,\txtHigh){\A};
          }
    \filldraw[ultra thick, draw=black, fill=green, opacity=0.2] (-4.7,1.8) -- (-4.7,8) -- (5.2,8) -- (5.2,1.8) -- (-4.7,1.8) ;
\end{tikzpicture}
\end{minipage}
\caption{Objective four definition}
\label{Paper Four}
\end{figure}

The use of statistical information for the management of complex organizations has evolved since the beginning of the XIXth century. Since the invention of population by XVIIIth century demographers[DESROSIERES], and the integration of numbers in the political and administrative language in the second XIXth century [PORTER], multiple types of information have been used for the orientation of public policies and the administration of public services. Meanwhile, the rise of epidemiology and the critalization of a body of knowledge around the institutions in charge of the defense of Public Health helped creating a specific Public Health oriented spin on quantitative information for health systems.

The use of data for policy making is a combination of data sources, statistical methods, and political or social norms, that will define the conditions of utilisation of statistical evidence for policy making. Finding the proper data source, being able to analyze it and incorporating the results of this analysis in a political process is essential to the proper use and utilization of information systems. In this regard, Alain Desrosières has shown how two traditions have been cohabiting in the early ages of the production of social statistics\cite{admin_savant}.

%In Global Health, the use of data for the definition of \textit{evidence based} intervention and policies has emerged as a panacea for project design and management. There are nonetheless difficulties in this regard. The global nature of public health means that statistical data available for analysis is by nature scattered and varied.

%reductio ad M\&E

\begin{quote}
The first tradition is administrative, and is based on political science and the law, on the German Staatenkunde, from the time of Conring and Achenwall. It is more taxonomic than metrological: it is designed to classify facts systematically rather than measure them, which is the essence of the other tradition, the "English" tradition. The latter, inspired more by the natural sciences and by progress made in measurement and probability theories, is a distant relation of the English political arithmetic of Graunt and Petty.
\end{quote}

Desrosières later shows how these two traditions have bee reconciled in the modern figure of the statistician, at the same time administrator and scientist. It is useful to keep considering this tension when thinking about maturing statistical systems like HMIS. Being able to distinguish between situations when actors of HMIS are acting as administrators, and when the position is that of a metrician is essential to understand HMIS issues and offer informed solutions.

This distinction is essential at many levels. The whole debate around the level of uncertainty that is bearable around a measurement is not only important for statisticians. Choosing a given approach will have an impact on how primary data will be collected, how it will be analyzed, and how it will be used. In many usages of HMIS, complete enumeration is deemed necessary, but this can be discussed. What is the level of confidence one can bear around the estimation of a stock of drugs ?

In sub-Saharan Africa, this tension is reinforced by a political tradition that has been structured around the colonial state. The structures and political traditions coming from this specific have complex relationships with the notions of uncertainty and control. Moreover, these structure are reenacting the colonial culture of exogeneous power structure, through that international actors take in the governance of African country.

This last paper will aim at understanding how some program managers in Mali the data available in HMIS, and how it impacts the way they think, design and implement HMIS programs. We will interrogate the notions of uncertainty, sampling, control and norms for this managers, and their appreciation and use of numerical evidence.

UNFINISHED

Reflections on social conditions of HMIS data usage  / politics of administrative statistics.

Data is not produced to create knowledge, but to implement disciplinary monitoring. Thinking mainly in terms of indicators.


    Figure \ref{Gantt4} will present a timeline for the realization of this objective.

    \begin{figure}[h]
    \begin{ganttchart}[vgrid,hgrid]{1}{24}
    \gantttitle{2016}{12}
    \gantttitle{2017}{12} \\
    \gantttitlelist{1,...,12}{1} \gantttitlelist{1,...,12}{1}\\
    %\ganttgroup{Group 1}{1}{7} \\
    \ganttbar{Data Extraction}{1}{3} \\
    \ganttbar{Data Cleaning}{2}{4} \\
    \ganttbar{Data Analysis 1}{5}{6} \\
    \ganttmilestone{Sharing First Results}{6} \\
    \ganttbar{Data Analysis 2}{7}{9} \\
    \ganttmilestone{Sharing Final Results}{9} \\
    \ganttbar{Paper Writing}{9}{11} \\
    \ganttmilestone{Paper Submission}{11}
    \end{ganttchart}
    \caption{Gantt Chart for Paper 4}
    \label{Gantt4}
    \end{figure}
