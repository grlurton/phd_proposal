\section{Introduction}

\subsection{General Overview}

Research in Global Health navigates between different levels of relevance. A global level, that aims at understanding global phenomenons and trends pertaining to the health of populations, and a local level, in which policies are designed, implemented and evaluated. The articulation between these two levels is essential, and gives Global Health its coherence and its specificity. Funding for local policies are defined by global institutions, using highly aggregated Health Metrics. In the meantime, global priorities are also informed by the results of local experiences, that are translated into generalizable practices.

A challenge for Global Health Metricians is to create and disseminate information that will be relevant at each of these levels. Defining metrics and approaches that will have global relevance and comparability seems to be a different work from creating actionable actionable evidence on which national and local deciders will be able to base their own decisions.

This challenge is not merely a technical one, but also a structural one. The field in which this information is produced is indeed not level, but bears the mark of historical and political structures, heritage from centuries of scientific development and nations building. The result of this history on public health expertise is described by Bergeron and Cassel as "a network involving many different actors, structures and tools, concepts and spatial and institutional arrangements" \citep{bergeron_savoirs_2014}.

Isn't the will to define a global research agenda to understand such complex and local equilibria the fundamental paradox of Global Health ? This question often finds technicist answers. The ability of statistics to offer rational basis to make decisions \citep{desrosieres_politique_1993,porter_trust_1996},
allied with progresses in data collection and analysis techniques haved fostered enthusiasm and hope in the ability of governments and other actors across the world to design and implement public health policies based on strong evidence \citep{abou-zahr_health_2005,shibuya_health_2005,bambas_nolen_strengthening_2005,mutemwa_hmis_2006,boerma_public_2013}. Better data, allied with better methods should  offer the basis to produce strong evidence, both at local and global levels.

In this regard, Global Health Metrics as a field has emerged as a set of tools and concepts to improve the quantification and comparison of health for every populations around the globe. Murray and Frenk described the field as unifying in its purposes as well as in its methods\citep{murray_health_2008}. The vision they delineate is one of standardized data, tools and methods, shared results and synthesized knowledge. Global projects such as the \gls{gbd} have demonstrated the success of this approach in fostering international cooperation and offering unified, comparable results on Global Health issues across time and space.

% "More broadly, the field of metrics and evaluation can serve several purposes: first, to sustain interest in and funding for global health by demonstrating positive results; second, to enhance efficiency by building a solid knowledge base of what works, thus generating a process of shared learning between countries; third, to improve the quality of decision making by providing sound evidence; fourth, to foster interdisciplinary dialogue by bringing together various areas of enquiry; and last, to promote the values of transparency and accountability as essential ingredients of democratic governance both nationally and globally."

% "Panel 2: Activities for the advancement of global-health metrics and evaluation
%• Development of new statistical methods and easy-to-use software to aid their application
%• Development of software and hardware to facilitate data collection through administrative records, surveys, or censuses
%• Setting global norms and standards for data capture, reporting and transmission such as the International Classification of Diseases (ICD) and the System of Health Accounts (SHA)
%• Increasing the availability of high-quality primary data.
%• Systematic analysis and synthesis of existing datasets to inform the public, research community, and decision makers
%• Strengthening the capacity to obtain, analyse, and use data through efforts such as those coordinated by the Health Metrics Network
% • Reporting and disseminating results to broad technical and non-technical audiences"

There is thus a pull towards norming and standardization in the production of Global Health knowledge, which is often understood as a technical necessity. Meanwhile, critical works in the field of global measurement have warned on the oversimplification provided by global indicators, and have questioned the ability of this approach to provide locally relevant information\citep{merry_measuring_2011,rottenburg_world_2016}. These critics should lead Global Health researchers to question the political assumption hidden in the methods we use, and to look into ways to mitigate their most stringent limitation. How can a global framework serve local action? In which conditions can a local metric be useful at a global level.

My work explores ways in which locally relevant health metrics articulate with global norms. I question this issue looking at different levels:

\begin{description}
\item[Critical Framework] My work relies on a critical approach to measurement in Global Health. I want to question measurement strategies that privilegy external validity and global comparability of metrics over local usability of the results. The main stake here is to provide a description of how the structure of Global Health as a political, economic and academic field heavily influences quantitative methods, by giving a higher priority to a global level over a local level. My goal is to offer a framework holding together the technical and the political characteristics of a Global Health measurement approach.
\item[Development] As a Global Health Metrics researcher, my main tool is the development of data processing and analysis methods that can be used to better understand data produced in low resource Health Information Systems. I question current methods and structures by looking at their limit of validity, and looking into how different approaches could yield better results.
\item[Implementation] Finally, the goal of this work is to be offer tools that could be used by practitioners in the field of Global Health. I work on the development and implementation of tools for daily usage. In the scope of this dissertation, this component will not be carried out, but each methods development has a potential future implementation as an end goal.
\end{description}

This variety of concerns in this dissertation is relevant as each dimension informs the others. By disentangling the technical necessity from the political expediency in measurement practice, I identify structures and practices that can be amended, once they are acknowledged. In the meantime, when developing new methods, I keep an eyes on how these methods can be used, and how they will be implemented.


\subsection{Work hypothesis}

My main work hypothesis relies on the distinction between a \textit{top-down} approach and a \textit{bottom-up} approach to measurement for health. The precise description of the characteristics of these approaches is one of the aims of this dissertation. In first intention nonetheless, I can describe \textit{top-down} methods as methods that are defined and standardized at global level, and later rolled out in different settings. This description fits most HMIS design recommendations for building Health Information Systems in developing countries. \citep{lippeveld_routine_2000,rhino_introducing_2003,daltilia_systeme_2005,health_metrics_network_framework_2008}. Moreover, it is the foundation of the \gls{me} field. Guidelines such as those of Global Health agencies like the Global Fund have important influences on the organization of health information systems in sub-Saharan African countries on which they are imposing definitions, classifications and reporting methods, to provide sufficient information for the evaluation of funded programs \citep{the_global_fund_global_2014}. The limitation of these approaches is that the asymetric relationship between global and local actors leads to a \textit{reductio ad \gls{me}} of information systems, that are built to answer imposed norms but do not create their own logics, cultures and traditions.

%The mindset behind these approaches is influenced both by the colonial administrative tradition, for which standardized statistical tools should provide a normed information, and by the developmentalist expertise tradition, along which an hyper directed data collection system is aimed at providing a unique and narrowly defined piece of knowledge. This mindset is still present in the current tendency of international donors to encourage the development of systems specifically aimed at answering their own particular information needs.

A contrario, \textit{bottom-up} methods are geared towards an aggregative approach of evidence. They rely on the use of data  locally collected in \textit{ad-hoc} ways. Most of the technical data work in this approach is dedicated to the upward movement of data aggregation, for example to produce national or international level analysis. This type of approach is \textit{de facto} used in health systems with a diversity of actors, for example when international NGOs use data systems that are different from the national Health Information System. It is often considered ineffective and dysfunctional, because of the difficulty to compare results from different local systems, and because of a perceived deficiency of health information systems that do not abide to validated international standards. Meanwhile, they have the benefit of fostering the development of local data culture, local data usage and ownership.

Of course, most projects do not fall completely in one of these categories. A project like the \gls{dhis2}, for example, aims at providing standardization of data at Health District level. Part of the theory behind this tool is that district level is the right level of standardization, to offer flexibility at facility level. On an other level the \gls{gbd} as a project has some of the aggregative characteristics of a \textit{bottom-up} approach, with some globalizing and standardizing characteristics of a \textit{top-down} project. Finding the right theoretical framework to make this classification operative will be one of the tasks of the third aim of this dissertation.



\subsection{Aims}

The principal aim of this dissertation is to test and describe the differences between \textit{top-down} and \textit{bottom-up} approaches to producing statistics in low resource health systems, and to explore the respective benefits of each approach. I address this principal aim through three specific aims. The two first aims will be concrete examples of analytic questions I will use as examples of the difference between the two perspectives. The third aim is to provide a generic framework justifying the relevance and the usefulness of differentiating between these two approaches to build more relevant health information systems.


\paragraph{Flexible Standards} Defining categories based on which people are going to be counted is an essential piece of the statistical work \citep{desrosieres_politique_1993}. It is an essential step in the simplification involved in the activity of measurement. The field of Global Heath relies on important taxonomies, like the International Classification of Diseases and on Metrics like the Disability Adjusted Life Years, to unify description and measurement of health across the globe, and allow comparison and benchmarking \citep{murray_towards_2007,murray_health_2008}. Meanwhile, at local level, the use of globally defined metrics may have its limits, as it does not allow adaptation to local contexts and situations. The measure of retention of HIV patients in care is an example of this. Understanding the outcome of patients after they enter care is essential to evaluating HIV care systems performance, and efforts have been made to track measure to inform strategic planning and programs evaluation   \citep{the_global_fund_global_2014}. Meanwhile, the measure used in most settings to track patients retention, namely the proportion of patients that are considered \gls{ltfu} after a certain time in care, is problematic. The high variation of  \gls{ltfu} rates between programs and the low specificity of this metric leads researcher to question the way Loss to Follow Up is defined and measured globally \citep{chi_universal_2011,yehia_comparing_2012,grimsrud_impact_2013,forster_electronic_2008}. Using \gls{emr} data, I will model how the measure of retention is affected by local contexts and data quality, and I will explore  more robust ways to measure retention in HIV care.

\paragraph{Data Hybridization} Unavailability of reliable data on population geographical distribution in a large number of countries is a well known issue \citep{mahapatra_civil_2007,mikkelsen_global_2015}. As a result, spatial distribution of populations, an essential piece of evidence to develop public health policies, is often unavailable at a policy relevant scale. To palliate these shortcomings, approaches for mapping of populations have been developed, involving the use of macro-level rasters of covariates such as land coverage or night lighting imagery  \citep{linard_population_2012,stevens_disaggregating_2015}. These approaches give interesting insights into how populations are distributed, and their output can easily be used for other public health work using the GPS coordinates system. Meanwhile, in situations with very low information on populations, the results of these top-down approaches is often little more than an overlay of covariates. Additionally, their results are hard to use for public health policy planning, as they do not link population to commonly used localization conventions such as places names. My second aim is to hybridize multiple data sources on population to produce a population map of Niger linked to the lowest level of population settlement possible.

\paragraph{Framework for Information Systems critique}  To understand how the distinction between \textit{top-down} and \textit{bottom-up} methods can help build  Health Information Systems in low resource settings, we need to build a critical framework to understand the \textit{Information Infrastructure} \citep{hunsinger_toward_2009} of these systems. I will offer an overview of the influences and assumptions that structure current health information systems in developing countries, and I will describe the enabling and limiting aspects of these systems, linking their technical characteristics to their political implications. Using results from the two first aims as well as other project I am working on, I will explore the relevance of the vertical classification of health information systems. Finally, I will delineate a framework in which practitioners can think the mix of methods they use in Health Information Systems, to serve their varied needs.

\subsection{Novelty and scientific contribution}

My dissertation is contributing to three main areas of research surrounding Health Information Systems. One is mostly investigated by the \gls{ict} community and interrogates the importance of local adaptation of information systems and its impact on the definition of standards. The second domain of relevance is more linked to the Global Health field, and on works that interrogate how data collected inside health systems can be used to inform decision making. The last domain of contribution will be more related to the \gls{sts} field, and will endeavor to integrate results from this field with applied statistical work.

\subsubsection{Local Adaptation and flexible standards}

In the ICT field, defining standards for data collection systems that can be implemented at local levels but respond to national or international norms. Jørn Braa, describing the approach that presided to the design and development of the \gls{dhis2} remarks that "the top-down and all-inclusive approach to standardization [is] common among ministries and central agencies" and pleads for \textit{flexible standards} following the idea that "the individual standards must be crafted in a manner which allows the whole complex system of standards to be adaptive to the local context" \citep{braa_developing_2007}. The need for local adaptability of Information Systems is seen as a key issue of \gls{ict} in developing countries \citep{macfarlane_harmonizing_2005,walsham_research_2006,walsham_foreword:_2007,jacucci_standardization_2006} and thus some \gls{ict} solutions have been designed and explored in the forms of tools like \gls{dhis2} or standards like the Open Health Information Exchange initiative.

Aims 1 explores ways in which standards for indicators definition or for the evaluation of data quality can be adapted to specific contexts. An extension of this work could also be the definition of methods for aggregation and comparison of metrics defined and measured using flexible standards.

\subsubsection{Imperfect data Usage}

Health Information Systems data is often underused, or not used at all by its intended users\citep{health_metrics_network_framework_2008}. This underuse is often blamed on the  perceived bad quality of primary data that would make it unfit for statistical analysis
\citep{ronveaux_immunization_2005,makombe_assessing_2008,heunis_accuracy_2011,gimbel_assessment_2011,who_assessment_2011,hahn_where_2013,kihuba_assessing_2014,glele_ahanhanzo_data_2015}. Approaches to solve this issue have mainly focused on improving primary data quality to improve data use  \citep{braa_improving_2012,mutale_improving_2013,ledikwe_improving_2014,nisingizwe_toward_2014}, but some authors have also pointed to the possibility to use even imperfect routine data to answer specific public health questions\citep{gething_improving_2006,gething_information_2007,wagenaar_using_2016}. These approaches rely on the nature of routine data, which is usually highly dimensional and structured times series. This type of data enables the use of robust modeling methods at an adequate level of aggregation, to control and correct for data errors.

My dissertation contributes to this line of research in two different ways.

Aim 1 offers improved insights in routine data quality and its impact on health indicators. Aim 1  explores a micro level modeling of data quality and its impact on the measure of patient retention. It will also define retention indicators less sensitive to data quality. Aim 1 and aim 2 also examine methods to enrich data, using \gls{emr} metadata or  hybridizing different data sources. Both approaches are an extension of usual approaches to health data, and make full use of the richness of modern data collection tools, and of the wide availability of public data sources.

\subsubsection{Internal Critic for health statistics}

Most of the critics made on the use of statistics to inform and manage public policies comes from a post-modernist perspective, with a tendency to put an emphasis on the shortcomings of statistical systems, and their hidden political agenda. Statistical tools are considered over-simplifying, biased, or sometimes just inadequate \citep{merry_measuring_2011}, and when used for Monitoring and Evaluation of public policies, they are considered too prone to enforce power relations instead of offering knowledge \citep{desrosieres_prouver_2014,gaudilliere_nouvel_2016}. My approach, as a statistician, will not be to try to prove or disprove these claims. Building on Latour's suggestion for a productive critic \citep{latour_why_2004}, I create a framework for the co-construction of a technical object and its critic. I will be exploring the relationship between social structures and analytical tools through the exploration of the technical limitations of these tools, and looking how their adaptation question current structures.

Aim 3 offers a framework in which to critically understand different approaches to health information systems, and anchors the development of new tools in this framework.


\subsection{Timeline}

Aims specific timelines are explained in more detail in sections \ref{timeline:aim1}, \ref{timeline:aim2} and \ref{timeline:aim3}. All the aims should be finalized and papers written by the end of the first quarter of 2018, which would allow for a finalization of the dissertation by June 2018.


\begin{figure}[!ht]
	\begin{ganttchart}[vgrid,hgrid,y unit chart=.6cm]{1}{9}
    \gantttitle{2017}{3}
    \gantttitle{2018}{6} \\
    \gantttitlelist{10,...,12}{1} \gantttitlelist{1,...,6}{1}\\

		\ganttset{bar/.append style={draw=red!40 , fill=red!40},
					group/.append style={draw=red, fill=red}}
		\ganttbar{Modeling and Model estimation}{1}{2} \\
		\ganttbar{Cohort Simulation}{1}{2} \\
		\ganttmilestone{Sharing Cohort simulation results}{2} \\
		\ganttbar{Data Quality Impact}{3}{3} \\
		\ganttbar{Data Maturity}{4}{5} \\
		\ganttbar{Robust Measures of retention}{4}{5} \\
		\ganttmilestone{Sharing Final Results}{6} \\
		\ganttbar{Paper Writing}{6}{8} \\

		\ganttset{bar/.append style={draw=green!40 , fill=green!40},
                    group/.append style={draw=green, fill=green}}
		\ganttbar{Name Matching}{1}{2} \\
		\ganttbar{Locality Mapping}{2}{3} \\
		\ganttmilestone{First complete map}{3} \\
		\ganttbar{Population Estimation}{4}{7} \\
		\ganttmilestone{Sharing Final Results}{7} \\
		\ganttbar{Paper Writing}{7}{9} \\

		\ganttset{bar/.append style={draw=blue!40 , fill=blue!40},
					group/.append style={draw=blue, fill=blue}}
		\ganttbar{Finalize Litterature review}{1}{4} \\
		\ganttbar{Finalize analytical Framework}{1}{6} \\
		\ganttmilestone{Seminar Presentation}{3} \\
		\ganttbar{Paper finalization}{4}{8} \\
	\end{ganttchart}
	\caption{Dissertation Timeline}
\end{figure}
