\section{Introduction}

If a literary form had to be chosen to write or talk about Health Management Information Systems (HMIS), the complaint would probably be everyone's favorite pick. Be it complaints on the burden of work involved in collecting, managing and analyzing data in health systems, or laments on the inexistence of good quality data in most developing countries health systems, HIS are usually described as a non performing burdens of health systems, that can only be improved [HMN citation]. This frustration has multiple causes, and is only matched by the expectations placed in HIS and their widely recognized importance, some authors calling HIS "the foundation of public health"\cite{foundph}. Collecting and analyzing information on activities and results of health systems and on the populations served is indeed essential to guide strategic decision making and to inform health policies.

The complexity of designing and operating well performing national HMIS comes from the fact that HMIS have to handle a high diversity of data and information. Meanwhile, producing this information requires the contribution of a multitude of actors, and is a huge organizational and methodological challenge. Data used to produce relevant health information may come from administrative records, organizational documents or population surveys, and are produced by a variety of actors and organizations, with differing cultures and approaches.
% TODO lister les différents types d'info à produire.
% BIBLIO HMN Network
% BIBLIO walsh07 in IS in PVD issue, 3 / 4 papers are on health

Finally, HIS have to be able to adapt rapidly to changing epidemiological, organizational or political situations. They have to be able to produce relevant information on emerging health issues, and to adapt to the entry of new actors or to a modification in the mode of management of health systems.

In richer countries, the issue surrounding health information is often one of regulation and standardization. The  existence of well performing and well established data sources on populations, and the relative ease of collecting massive amounts of data on individuals pose questions that are mainly related to the protection of privacy, and to the definition of standards for interoperability. The definition of what information should or can be produced is usually a legislative and matter, handled by dedicated entities.

In developing countries, where population data is scarce and data collection can be much more of a challenge, the issues are much different. Health policies depend heavily on the financial, political and technical support of international actors, with differing orientations and priorities. As a consequence, developing countries HIS strengthening programs can be classified into two great families : technology based solutions and institutional reforms. The first family is mainly targeted towards improvement of data collection and relies on solutions relying on the revolution in  data collection capabilities. The second family relies on the adequation of health information produces to policy makers needs, and usually promotes a top-down and normative approach to Information Systems.

If each of these approaches has its benefits and some success stories,  they appear to miss an important part of what makes information systems work. They put their emphasis to the extremes of the information production chains (cf. Figure \ref{HMISFunctions}) and undervalue the middle tiers of these systems. This proposal designs research program oriented towards exploring ways in which this middle tier can be mobilized to improve the value of information produced by HMIS.

This document will start by an introduction of the conceptual framework surrounding the proposed research. We will define Health Management Information Systems, using two classic schematic approaches. We will then define the objectives we will pursue in our research, and will finally describe our different aims and the methods we want to use to complete them.
