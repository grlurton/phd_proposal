\section[Understanding Information]{Aim 3 - Understanding information - what a bottom up appraoch is}

\subsection{Objective}
Global Health as a system of knowledge is situated at the intersection between a global and a local level of inquiry. The global level is the seat of international institutions, NGOs, academic structures or funding agencies, that work on addressing health questions framed as unified fields around the globe. This level aims at producing a generalizable and transferrable or translocal  body of knowledge and skills that form a global professional field \citep{merry_measuring_2011}. The local level is the seat of national and sub-national governments and administrations, civil society organizations and local actors, who focus on implementing health interventions to tackle localized health questions. The knowledge produced and used at this level aims for specificity contextual, used micro level management and short-term monitoring of programs.

For most actors of the Global Health field, making the connection between these two levels is an important part of their daily work. Local Governments will work on designing policies informed by research formulated and evaluated by the global academic community, and will appeal to international organizations to fund these local policies. International NGOs will have headquarters and local missions, who will negotiate the exact nature of their programs. The Global Fund to fight Aids, Tuberculosis and Malaria (GFATM) even have a complex and specific organization to make this connection, with a Country Coordinating Mechanisms (CCM) in charge of being the voice of local stakeholders to the Global Fund, and Local Fund Agents (LFA), in charge of enforcing global management standards in local programs.

This constant circulation between these different levels of action and bodies of knowledge is common to most fields of International Development professional. Rosalind Eyben described how an important skill of a development professional is her ability to juggle between the \textit{substantialist} system of knowledge of big development organizations, and the \textit{elationalism} she has to mobilize to interact with local actors \citep{eyben_hiding_2010}. Speaking the global jargon of logical frameworks and clear-cut causalities while navigating local messiness and discontinuities of local settings is indeed a unique challenge. Global Health actors have started to solidify the knowledges and skills needed to navigate these different levels, by inventing an Implementation Science, aimed at improving the capitalization of local experiences at global level, while facilitating the translation of this global set of evidence in local programs.

Meanwhile, the fluidity of the circulation of knowledge between these different levels should be interrogated. Eyben notes the unequal power structure between the two systems she describes. What she describes as the substantialist system is indeed part of the dominant approach to modern management, sometimes described as the neo-liberal approach of governance, focused on measurable results and impacts, which drives the implementation and financing of most Global Health projects . Building an approach of knowledge at the core of the economic and political governance of the field generates an asymmetry in a potential exchange of methods and results. Another driver of this asymmetry is the unbalanced legitimacy between local and global actors when it comes to knowledge production. Global institutions are driving the definition of what should be known and how it should be known, by setting the research agendas, training scholars and altogether shaping the way knowledge should inform policies . Finally, we can not underestimate the legacy of colonial knowledge systems. Arjun Appadurai  has described how these systems, geared towards providing a standardized vision of colonial domains and of their populations to help their administration and control more than their deep comprehension, still shape the vision of how global knowledge can be created \citep{appadurai_number_1996}.

This asymmetry in the production of knowledge is especially striking when it comes to define and gather statistical evidence. In a short text on the development of statistical measurement systems in developing countries, Alain Desrosières notes that unlike Northern countries where “the specialization of work has routinized the work of statisticians, encapsulating methodological standards that are seldom re-scrutinized” , developing countries have much less pre-established system, thus leaving space for methodological innovation. Meanwhile, the “invention” of these statistical systems happens in a top-down fashion, that considerably orients how this innovation happen. “A common feature [of examples of statistical innovations in the developing world] is that they are situated at the top of the chain of production and utilization of statistics, that is to say at the moment of the conception of the data collection systems that will be implemented to produce these statistics (…). These discussions are held among economists rather than among statisticians, and are about the “choice of indicator” rather than on the data sources and methods that will be used to measure them (…). The actual quantification thus appears to be secondary to the “choice of indicators”, even when experience from specialized researchers in Development studies show how much this phase impacts the way these indicators can be interpreted. This is even more true in countries where conventions and categories are less established and routinized.” \citep{desrosieres_pays_2014}

In the specific context of Global Health, we can find the main features of this description in what can be described as a top-down organization of statistical systems. The categories of measurement are defined through Monitoring and Evaluation frameworks, aimed at providing information to funders and global program managers. To ensure the renewal of grants, and to be considered as legitimate Global Health knowledge, \gls{me}frameworks have to be organized around specific lists of indicators, with imposed methods and schedules. In this organization, the categories and indicators are defined at the global level, without consideration of how data collection is happening at the local level. As such, the qualities of good measurement categories are defined on common criteria used in Epidemiology and Public Health, rather than on their measurability or on their adaptation to available data. In the meantime, the goal of comparability of indicators at the global level supersedes the need for adequacy and validity at local level. The dynamics of statistical systems are guided by the definition of long lists of indicators, rather than by the definition of specific measurement frameworks from which to derive what information is usable. The asymmetry between global and local levels of knowledge is here evident, as the conditions for the definition of local methodological research and innovation are negated from the inception of statistical development projects.

To better understand how this top-down system operates, we need to look into a specific example. HIV as a field of Global Health is exemplary when it comes to thinking about strategic information collection. First, HIV has been the field in which a lot of the contemporary Global Health practice has been invented. Be it in the involvement of multiple stakeholders at both global and local levels, in the implementation of results oriented management of global programs or in the creation of a global community of knowledge and practices, HIV has been a laboratory of Global Health practices . Second, in many health systems, HIV has been the entry point of large scale chronic disease management. As such, radical methodological innovation has been necessary to measure the implementation and results of public health programs for chronic disease. Looking into the result of this innovation is a perfect reduction of the dynamics at hand in the creation of measurement systems in other domains of Global Health.

While looking at how measurement is thought of and made in HIV programs, this research project has two main objectives. First, I explore the conditions of local measurement systems for Global Health. The reliance of Global Health on global standards of measurement is indeed widely considered as an essential feature of the field. Global nomenclatures like the ICD10 and global measurement frameworks like \gls{me}standards seem to speak to the ideal of a unified field of measurement towards which each actor of the field should aspire. Meanwhile, the importance of the creation of local statistical systems, deeply embedded in local administrative and institutional frameworks has been thought, since the late modern period, as an important feature of building modern states. French revolutionaries like the Abbé Siéyès considered adunation of the territory, by which they meant the unification, on the territory, of administrative and measurement systems, as a necessary condition for the creation of a well performing national statistical system \citep{alain_desrosieres_prefet_1993}.

The second objective of this work is to offer an internal critic to current practice in measurement. The creation and use of quantitative evidence in public policy have often been critiqued as a mean of creating a manageable reality for governments, while simplifying, norming and enframing the rationality of actors. This critique is important, but presents quantitative researchers with little alternatives to improve their methods, and ultimately are limited in their ability to explain the mechanisms at play in measurement . By approaching the methods, their usage and their misuse in non-technical framework, these critics tend to posit the impossibility of statistics as a productive and convivial tool for government. In this work, I try to show how quantitative analysis can complement these critiques to improve the comprehension of the dynamics at play, and offer an improvement of existing quantitative tools.

%As a field, Global Health has to navigate between different levels of relevance. A Global Level, that aims at understanding global trends and phenomenons, and a local level, in which policies are implemented. In this regard, one of the challenges of Global Health is to create and disseminate information that will be relevant at these different levels. The difficulty in doing this is not merely a technical one, but also a political and structural one. The field in which this information is produced is indeed pre-organized by historic and political structures, that are not adapted for this multi-level adaptation of knowledge. National Statistical systems, for example, are the products of the larger systems they are designed to serve. As such, Health Information Systems are the results of the Health Systems in which they are created, and are influenced by the administrative cultures and the political context they are supposed to inform \citep{bergeron_savoirs_2014}.

%This relation between local and global levels of Global Health is nonetheless asymmetric. As a scientific field, Global Health is essentially a hierarchical organization, where political, academic and economic institutions situated in high resource countries world offer guidance and validation for the methods and results used to measure health over the world. As a result, the field has to find an equilibrium between a downwards  \textit{standardization} of information and an upward \textit{aggregation} of locally produced data.

%The social background in which this equilibrium is sought is important and has its roots in the origins of health information system in developing countries. Whereas European statistical systems development was led by social activists who guided the design of the first modern welfare systems \citep{desrosieres_politique_1993,desrosieres_administrator_1997}, colonial statistical systems were geared towards efficient land administration and economic exploitation \citep{rambert_cartographie_1922,de_martonne_cartographie_1931} with little attention to local population. The development of these statistical systems can indeed be traced to the enforcement of a specific mode of administrative control by colonial powers in the XIXth century \citep{appadurai_number_1996,cordell_couting_2010,gervais_how_2010}.
%Colonial statisticians, often weakly skilled or trained \citep{kateb_gestion_1998,cordell_couting_2010}, nonetheless set the nomenclatures and conventions around how land and populations would be described and analyzed \citep{rambert_cartographie_1922,gervais_how_2010}. Enumeration and its categories did not emerge to describe local complexities and specifities, but were on the contrary simplifications aimed at creating a uniform colonial subject, to which a standard colonial rule would apply \citep{said_orientalism_1979,appadurai_number_1996}.


%This historical background can be prolonged after the decolonizations, and can present some continuities with the way public health statistics are produced in developing countries today. Understanding how this longstanding history affects the way people involved in Global Health Metrics think about the production of quantitative evidence is important for practitioners of Global Health Metrics to understand the limits of their work, and the improvements they can bring.
% In these settings, strong public health information systems are the result of local equilibria and of adaptations to local statistical cultures, started by individuals making ad-hoc use of different data sources and inventing and standardizing methods on the run. \citep{lecuyer_medecins_1987}.

% The aim of this paper will be to provide a clear understanding on how the structure of the field of Global Health can have an influence on how health metrics are produced, and how they are being used, or not used. Providing this framework on the conditions for designing quantitative evidence in Global Health is essential to design useful methods and metrics that can help shape public health policies at local and global levels. I will then use this framework to show how it can help understand and bring into context the results of my applied work.

%The emergence of Global Health Metrics as a field is aimed at providing "valid, reliable and comparable measures of the health states of individuals and of the health status of populations" \citep{mathers_population_2003} This triad of characteristics may be hard to ensure in a single scope, especially when the locally relevant may not be globally comparable. Meanwhile, this paper will explore the hypothesis that the emphasis given to comparability in this triade is historically, intellectually and politically situated, and can be linked to a culture of top-down standardization of knowledge in the developing world.


\subsection{Method}

This paper will build on the work and results of the two previous aims as well as on other work I have done in the field, to understand the conditions in which a \textsl{bottom-up} approach to health metrics can be defined and put into practice. The methodological challenge of this to work is to enter the domain of critique from within my own field \citep{latour_why_2004}. Most of the critical work on the use of quantitative evidence in health policy indeed comes from researchers that are not practitioners of Global Health Metrics \citep{merry_measuring_2011} . In this regard, even the most nuanced critique suffers the temptation of an overwhelming challenge to the possibility of quantitative work\citep{latour_why_2004}.

I will thus offer a framework and examples for an internal critique of this work. Using the opposition between a downward, normative movement and an upward, aggregative movement of methods and data, I will offer a simple explanation of how power relations that structure Global Health Metrics can be found in the methods that are used, and I will show how understanding the limits of these methods can help challenge this power structure, to reequilibrate the relationship between local and global actors of Global Health.

This work will be built in three main parts:

\begin{enumerate}
\item Provide a description of how a top-down approach to Health Metrics differs from a bottom-up approach.
\item Delineate how the structures in place in Global Health can influence which of these approach will be used by Global Health actors.
\item Describe the comparative benefits and shortcomings of each approach, and how they can be combined.
\end{enumerate}

My main hypothesis is that the top-down development of statistical systems can be traced through two main channels:
\begin{description}
\item[Methodological transplantation -] the methods used at the local level are often partial adaptations of methods developed by global actors. This is akin to a transplantation, as the context in which methods are developed differs often importantly from the context in which it is implemented. This could be because of differences in infrastructures, or because of the transplantation between an academic setting and an administrative setting. On the other hand, this imposition of methods and practices limits the ability of local systems to develop their own sets of practices, and allows only limited adaptation or derivation of methods.
\item[Imposition of Nomenclatures - ] in order to ensure comparability in time and space, nomenclatures of categories are imposed on local levels with little margin for adaptation. The strength of standards meanwhile mainly benefits to the global level, and is really a displacement of the computational work to unskilled and unspecialized actors. The effectiveness of the norm is meanwhile subject to caution when one observes real computation practices.
\end{description}


The main methodological challenge of this paper is to make a common reading of literature from \gls{sts} and Global Health Metrics.  I organize this project in four different parts:

\begin{enumerate}
\item A documentary analysis of \gls{me}frameworks for HIV currently in use, to identify how methodological transplantation and nomenclature imposition are currently playing in existing programs.
\item A review of the \gls{sts} literature pertaining to statistical systems development and their framing.
\item An update of the results from my technical work, to show how a bottom up approach to statistics is possible.
\end{enumerate}

The update of the results from my technical work has two main aspects. I want to give a good description of how the methods I use to improve retention metrics, or to mobilize local demographic data, can be used to challenge methods transplantation, and to interrogate the relevance of global nomenclatures. Meanwhile, this questioning will have to be completed by a description of how the conditions for a better collaboration of global and local levels can be fostered. I will use a framework proposed by Jørn Braa in 2007 for what flexible standards for Health Information Systems can be \citep{braa_developing_2007}. Braa differentiates two levels of modularization of standards. A vertical modularization, which “corresponds to traditional layering in software engineering where one layer offers services to the layer above. Separate standards are defined at each layer”, and a vertical modularization which “means that rather than going for one “universal” standard for a domain, one makes several standards—one for each part of the domain—and interfaces are defined between them.”. These principles, used to build scalable and adaptable software, can help to design an aggregative model of knowledge in Global Health. By fostering the relevance and unity of local statistical analysis, I will complete my work by describing how this local analysis can be aggregated at a global level, or compared between settings.

\subsection{Timeline}
\label{timeline:aim3}

I have already made most of the literature review for this work. A first paper on this work has been accepted for presentation in a workshop at the Ecole des Hautes Etudes en Sciences Sociales. Another presentation of this work for social scientists interested in Data Science should be held at the UW in december. This will allow me to get feedback and to complete my analytical framework. Most of the remaining work for this aim is to include the results from the two other aims and to situate them in my analytic framework.


\begin{figure}[!t]
	\begin{ganttchart}[vgrid,hgrid,y unit chart=.6cm]{1}{9}
		\gantttitle{2017}{3}
		\gantttitle{2018}{6} \\
		\gantttitlelist{10,...,12}{1} \gantttitlelist{1,...,6}{1}\\

		\ganttset{bar/.append style={draw=blue!40 , fill=blue!40},
					group/.append style={draw=blue, fill=blue}}
		\ganttbar{Finalize Litterature review}{1}{4} \\
		\ganttbar{Finalize analytical Framework}{1}{6} \\
		\ganttmilestone{Seminar Presentation}{3} \\
		\ganttbar{Paper finalization}{4}{8} \\
	\end{ganttchart}
	\caption{Gantt Chart for Aim 3}
	\label{GanttPaper3}
\end{figure}
