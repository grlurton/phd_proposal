\section[Innovative analytic approach - data integration for malaria elimination]{Innovative analytic approach - data integration for malaria elimination\footnote{This Section is currently the adaptation of a grant proposal submuitted for this aim}}

\subsection{The challenge of local population mapping}

Mapping the target population of health interventions is key to their planning or evaluation. Meanwhile, the absence of civil registration or other form of precise population registries, it is often hard to estimate the size of the population living in a given area. Some work aimed at producing local estimations of populations uses a top down approach to this problem, projecting high level estimations of populations, and defining projections of this aggregate number, as informed by multiple layers of covariates, such as geographic features or nighttime light. This approach, if it gives interesting insights into probable distributions of populations, end up displaying mainly a overlay of covariate layers, when no middle level or local population data is available.
% TODO add illustration for Niger.

Moreover, if this top-down approach is useful to provide macro-level perspectives on population distribution, it is not useful for local level use. Local actors typically think in terms of places names, more than in terms of GPS coordinates. Additionaly, attractivity of a urban center for rural population is hard to model through macro level covariates, as it often depends on such factors as tradition, habits or administrative border drawing.

\subsection{Voters registration data as a demographic data source}

A data source that is, to our knowledge, seldom used to inform population mapping for public health purposes, is voters registration lists. There is meanwhile a case to be made for the use of voters' registration data to estimate size of populations. By definition, voters' registration should aim at being as complete as possible a register of adults in the nation. Moreover, in most democracies, some form of national elections are held at least  every five years, leading to an update at least partial of voters' registrations. In sub-Saharan Africa, between the years 2015 and 2016, 27 countries were supposed to hold national elections, leading to a theoretical registration of XX\% of the adult population of the continent. Finally, for transparency and accountability reasons, electors registries are usually supposed to be easily accessible.

Due to the sensitive and political use of these data, the quality of voters registries are often described as non trustworthy. On other hand, for the same sensitivity reasons, one can argue that little data sources used in common global health practice are more scrutinized and criticized than voters registries.

\subsection{Data}

\subsubsection{The Niger 2016 elections voters registry}

\subsubsection{RENALOC}

\subsubsection{OpenStreetMap}

\subsection{Methods}

\subsubsection{Data Mapping / Name resolution}

In order to map the Niger voters' list, we need to map voting bureaux to geolocalized entities. We will use two different data sources for geolocalization, both partial and to be completed. The RENALOC, indeed is very complete, but appears to be faulty in terms of localization. OSM, on the other hand, can be considered gold standard in terms of geolocalization, but is less complete than the RENALOC. Combining those two data sources, meanwhile, we will aim at producing an almost complete map of Niger localities.

Map completion will be done in two steps :


\paragraph{Name Resolution}

\paragraph{RENALOC correction}



\subsubsection{Population estimation}

\subsubsection{Age distribution estimation}

\subsubsection{Expected result : population access api}
