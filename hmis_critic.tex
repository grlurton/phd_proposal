%%%%%%%%%%%%%%%%%%%%%%%%%%%%%%%%%%%%%%%%%%%%%%%%%%%%%%%%%%%%%%%%%%%%%%%%%%%%%%%%

\documentclass[letterpaper, 10 pt, conference]{IEEEconf}  % Comment this line out
                                                          % if you need a4paper
%\documentclass[a4paper, 10pt, conference]{ieeeconf}      % Use this line for a4
                                                          % paper

%\IEEEoverridecommandlockouts                            % This command is only
                                                          % needed if you want to
                                                          % use the \thanks command
%\overrideIEEEmargins
% See the \addtolength command later in the file to balance the column lengths
% on the last page of the document
\usepackage[english]{babel}
\usepackage[utf8]{inputenc}
\usepackage[T1]{fontenc}


\usepackage[numbers]{natbib}
\bibliographystyle{ieeetr}


\title{\LARGE \bf
Health Information Systems : not a foundation
}

\author{Grégoire Lurton$^{1}$ % <-this % stops a space
%\thanks{*This work was not supported by any organization}% <-this % stops a space
%\thanks{$^{1}$H. Kwakernaak is with Faculty of Electrical Engineering, Mathematics and Computer Science,
%        University of Twente, 7500 AE Enschede, The Netherlands
%        {\tt\small h.kwakernaak at papercept.net}}%
%\thanks{$^{2}$P. Misra is with the Department of Electrical Engineering, Wright State University,%
%        Dayton, OH 45435, USA
%        {\tt\small p.misra at ieee.org}}%
}


\begin{document}



\maketitle
\thispagestyle{empty}
\pagestyle{empty}


%%%%%%%%%%%%%%%%%%%%%%%%%%%%%%%%%%%%%%%%%%%%%%%%%%%%%%%%%%%%%%%%%%%%%%%%%%%%%%%%
\begin{abstract}

	Abstract here

\end{abstract}


%%%%%%%%%%%%%%%%%%%%%%%%%%%%%%%%%%%%%%%%%%%%%%%%%%%%%%%%%%%%%%%%%%%%%%%%%%%%%%%%
\section{Introduction}

In 1891, pioneer statistician Francis Galton wrote : "It is a human frailty to which statisticians are eminently liable, to look upon means as ends. They learn to take keen pleasure in the mere accumulation of neatly tabulated figures, carefully added and averaged, quite irrespectively of any use to which those figures can be applied.". \cite{galton_useful_1891}. If Sir Galton was at the time writing about the production of anthropometric measures in american universities, his remarks ring as extremely relevant today, we the call to produce, publish and disseminate information has been revived by the ease of the process.

The need to produce and publish high quality information is now recognized as inevitable for most modern organizations. Authors have recognized the early role of statistics in the invention and installation of democratic governance \cite{porter_trust_1996}, and their importance for the management and strategic governance of most organizations. The faith in the capacity of the use of numerical data to empower individuals and organization to reach their full potential in the developing work has even led some to declare the advent of a \textit{Data Revolution} \cite{independent_expert_group_on_a_data_revolution_for_sustainable_development_world_2014, center_for_global_development_delivering_2014}.

This faith in data as a mean of government, meanwhile, and as noted by Sir Galton, sometimes has a hard time differentiating means and ends.

In the Global Health field, the importance of Health Information Systems (HIS) have been claimed, and these systems have been deemed essential to proper decision making and administration of health programs \cite{abou-zahr_health_2005}. Meanwhile, the weakness of health information systems, and the inadequacy of current systems in a lot of  developing countries is widely recognized, and has fueled the development of a dedicated academic literature genre \cite{abou-zahr_better_2010} \cite{kiberu_strengthening_2014}. Meanwhile, the health sector shouldn't be too embarrassed on the state of its information generation capabilities, as other fascinating bodies of works describe the weakness of information provided in other sectors of public life as well \cite{jerven_poor_2013}.

Meanwhile, this literature, if it questions the performance of HIS, does not question the notion of their possibility. The existence anywhere in a the world of a coherent, purposive and purposively designed systems producing all information relevant for the strategic planning and daily management of health systems is nonetheless far from evident. It is often unclear even for practitioners, what HIS are or are not. As systems, health information systems are indeed recognized as a complex ensemble of tools, methods and processes aimed at answering a diversity of information needs for actors in charge of a multitude of decisions in health systems. Meanwhile, defining what should be happening in HIS and how it should happen is not as straightforward as it may appear.

In the effort to provide guidance and advice on how to produce much needed information for health systems, Health Information Systems are presented as unified systems, with inner logics and methods. Frameworks are provided, that present guidance and advice on how to design and administer HIS \cite{health_metrics_network_framework_2008}. These documents are helpful for practitioners and field workers, but there is an inherent risk in these normative approaches, which is to provide an illusion of unity, coherence and inevitability in some systems. The definition of requirements both in terms of purposes and tools of HIS results lowers the incentives to problematize this information and to limit the role of HIS strengthening to answering these requirement more than using local and specific assets and problems to find local solutions. In other words, once a standard is defined, all questions asked only appear to be technical questions on how to achieve and enforce this standard, to the detriment of problematization and customization of these standards. rdsj

knowledge and expertise are "a network that connect all sorts of actors, systems, tools, concepts and institutional and spatial arrangements." \cite{bergeron_savoirs_2014}

There is a temptation to group this complex knowledge into unified systems, even if they are made complex, for tractability and probably operationnal usability \cite{health_metrics_network_framework_2008} \cite{vital_wave_consulting_health_2009} \cite{daltilia_systeme_2005} [HMN , Vital Wave , wodon]. Meanwhile, the production and use of information a population, its health and the health systems it has access to indeed has a long and varied history, with different traditions and intellectual origins. The concentration on short term interventions or projects, aimed at strengthening specific aspects of these systems fosters a loss of perspective when it comes to understanding the globality of a system. This is especially true in countries with limited national governemental resource and capabilities, where it is even harder to enforce a unified approach to producing this information. In many settings, the repeated design of strategic information plans and other M\&E plans is happens in the meantime as the implementation of atomised and unstructured subsystems, and the piecemeal collection of  evidence and data by a multitude of actors.

Understanding the differences in premices and historical precedence of different approaches appears essential in empowering actors to understand and objectify choices, shortcuts and traditions, and to imagine innovative approach.

There are indeed multiple mindsets and traditions when it comes to the quantification and quantification of public health. These are generated from the history and traditions of statistics both as a scientific and an administrative field. The field of public health has also generated different traditions and approach. In former colonies, these traditions and mindsets have been compounded by the colonial experience, which created its own set of values and practices when it came to population statistics and the use of evidence for administration. Finally, the emergence of new technologies and the generalisation of powerful computing methods have deeply modified the way statistics are thought and made, and this in turn put a final layer of complication in the way health information is considered in the developing world.

This paper will offer a quick perspective into long term trends that ended up producing contemporaneous HIS questions. We will do so building on a litterature close to the Science Studies movement and Colonial Studies. We feel this body of research is especially important in the field of Global Health, in which the time frame is often the time frame of the project or the planification round, and where questions are rediscovered and solutions reinvented with often an historic short-sightedness, with little perspective or critical insigth in previous practices.

From this review, we will offer an approach to questioning HIS related intervention, and to understand how they can contribute to fostering evidence generation and use in developing countries.

\cite{vital_wave_consulting_health_2009}. Collection of methods and traditions from different fields to create multi-morphous knowledge. \cite{bergeron_savoirs_2014}

\section{Trends in information systems}

We will quickly discuss three main dimensions of the evolution of health information systems in developing countries. The first dimension is the mechanisms that lead to the development of a public statistics as a field, and more specifically how has the field of health information emerged. The second dimension is an approach to how public statistics were understood and produced in the colonial world. Finally, the question of how technics and methods innovation affect HIS will be approached.

\subsection{Health Information Systems as public statistic systems}

A first aspect we want to point is the fact that there is an exaggeration of how governments or central bodies can monitor and plan how statistical information can be produced and distributed.

Ian Hacking pointed that statistics, not like other forms of knowledge, have been developed to specifically serve political actors [HACKING]. Authors usually point that the term statistics is, etymologically, linked to the state. The use of information for public decision and policy making emerged developed concurrently in the health sector and to inform decisions in the economic, education and justice domains \cite{desrosieres_politique_1993, lecuyer_medecins_1977, porter_trust_1996}.

In the health sector, the claim is that "reliable and timely health information is the foundation of public health action" \cite{abou-zahr_health_2005, health_metrics_network_framework_2008}. The mobilisation and usage of information is indeed part of the genesis of public health, with tutelary figures like John Snow, Florence Nightingale and William Farr England, and the development of population level studies the French hygienists movement	\cite{porter_trust_1996}, and more specifically the figure of Villermé who 'creates the fusion between hygiene, statistics and the study of social world and its evolutions'.

Even in countries where public health and statistics emerged in and were developed, no centralized Health Information System to inform all health related decision was in place. A reason for this may be the high variety of subjects that can be considered part of a Health Information System. In France, From its initial development, the \textit{Annales d'hygiène} contains articles that touch to the diversity of interests of public health : general demography, epidemiological statistics, study of healthcare institutions, and study of different social and environmental problematics \cite{lecuyer_medecins_1977}. Meanwhile, most of these studies came from studies led by individuals, or from the secondary usage of administrative or other data.

There is also a false sense of unity in the definition of Information Systems, as what constitutes the needs for information for government is culture dependent. Two main historical archetypes are classically described. The German tradition is by nature descriptive, and aims at an exhaustive description of  populations and territories. More litterary than numerical, this tradition is nonetheless the one where the term \textit{Statistik}, derived from the root \textit{Staat}, the State, was coined. The mental framework behind this approach is thus that an exhaustive knowledge of the nation is necessary for the state to govern effectively.

This German tradition is opposed to an English tradition, characterised by the inferential approach of Graunt and Petty. The English approach is less descriptive and more inferential than the German and is more concerned in the estimation of aggregate indicators, and the search for correlations  \cite{desrosieres_politique_1993}. Figures like John Snow or William Farr have been trying to answer precise questions, regarding the cause of cholera epidemics or the differences in mortality in hospitals, using a host of different data sources, and combining them to reach a final answer.

These cultural differences in the approach to the use of statistics for government are also mediated in the medical field by different approach to health and medical interventions. In France, epidemiology as a field has historically been more oriented towards mathematical modelling while the US tradition was more oriented towards action and the evaluation of health interventions \cite{bergeron_savoirs_2014}.

Even with  of approaches and method is from a very early point the result of an international cooperation between statisticians. Statisticians with different approaches would exchange their works and their approaches. In the \textit{Annales d'Hygiène}, contributors from XXXX to YYYY were an international crowd and came from various countries, exchanging ideas, results and methods \cite{lecuyer_medecins_1977}. On an international level, statisticians would structure as a profession, with an international organisation to guide and influence practice	\cite{desrosieres_politique_1993, desrosieres_administrator_1997}. Moreover, really quick, statistician was a good choice of professional carrier, even for medical professionals, which allowed good scientists to participate and contribute in this developing fied \cite{lecuyer_medecins_1977}.




%Foucaldian perspective

\subsection{The heritage of colonial statistics}

When building health information systems in the developing world, one should not underplay the impact of the colonial heritage in the structures and institutions of current states. It is also true that the development of statistics in former colonies has conducted in a very biased and specific fashion.

Statistics in the colonies was indeed an act of control as much as an act of knowledge \cite{appadurai_number_1996}. In Europe, influence of social statistics for welfare design. Minor role in colonial states \cite{cordell_couting_2010}

"In itself, the directive amounted to a document of many hundreds of pages, mandating sixty-five different kinds of tables. The model and framework for the tables were accompanied by demands for very detailed data, along with dates for completing their collection. In addition, the tables had to correspond to numbered sections of the annual reports describing the activities of individual administrative units. The only think not prescribed in this demented document was how these tables, formulated in Paris, would be completed and by whom" this description of the 1909 directives for the collection of statistics in French colonies is made by Albert Ficatier, who was later involved in the reform of this system and in the development of new statistical systems in the newly independent countries.

Colonial statistics was using rudimentary estimates, compiled by less trained administrators and outside observers (Cite Cordell Ittmann  Maddox) with unclear incentives (Gervais and Mandé). Only in the 30's are methods converging with the ones used in the metropole.

Space definition and toponimy influenced by colonial classification and choices (\cite{gervais_how_2010} + article histo carto). This set the framework for later denumeration by setting cuonting units. Lasting impact.

And then Global Health. African populations as object of external description, Cite Bonneuil

\subsection{Methods innovation in public statistic systems}
%OpenHIE

Development of EMR

ABSENT = outillage Statistique. = > Le tournant de Global Health Metrics ? \cite{health_metrics_network_framework_2008} promotes global frameworks and norms.
Paper vs computer debate

Sampling and then big data. New methods.

BEsoin de s'adapter à Big Data



\section{A typology of HIS interventions}

Understanding the long term trends in thinking and producing information on the health of population and health systems is important for organisations and people currently working health information systems strengthening projects, in order for them to understand their positioning and implications. Long term phenomenons are indeed still present in the way we think about how to improve evidence generation in health systems. There are multiple ways in which health infroamtion systems intervnetions in developing countries are being made. We hereby offer a simple typology of these interventions, and trace each intervention type to corresponding frameworks.

Taking in consideration a variety of Health Information Systems intervention in developing countries, we differentiate three main approaches in these programs. In this exercise as in a lot of situation, there is of course no intervention that can be considered pure type, and any project will be using forms and practices from different approaches. Meanwhile, understanding when and why different solutions are geared from different mindset appears essential for understanding the promises and shortcomings of different approaches.

\subsection{Process oriented Approach : the puzzle approach}

A first type of interventions is putting emphasis on the systemic dimension of HIS, and focuses on strengthening processes and organisation of HIS. This is a mainly goals oriented approach, in the sense that the starting point of these interventions is usually the defeinition of information needs for administrators or end users of this information.

We call this appraoch the jigsaw-puzzle approach, because of the way it provides information. From an overarching image that one wants to reconcile, pieces have been cut and have on and only way to be assembled to provide the picture we want to see. In this sense, the jigsaw puzzle is very much a social and organized experience (in the words of Georges Perec : "puzzling is not a solitary game: every move the puzzler makes, the puzzle-maker has made before")

This approach will put an emphasis on the use and institutionnal usage of information. A weakness of such interventions is their rigidity, and their heavy determinism. Producing normative docuemtns, frameworkds and guidelines and gearing for standardization.

Multiple projects can use this approach. RHINO, HMN, M\&E. Some patients files design \cite{health_metrics_network_framework_2008, rhino_introducing_2003} .

\subsection{Data Collection Approach : the pixel approach}

A second type of projects and interventions is putting great emphasis on the collection of data. These approaches are usually relying on more technical perspectives. ODK , OpenMRS,

This approach is built on the belief there is a fundamental value of data collection, and more data is better, and information systems are first and foremost data collection tools that have to be optimized and should perform. These approach can be fostered by programs that value first and foremost individual patient care,

Usually computer based, but not always. Risk is the undervaluation of politidal determinants.

\subsection{Computing Approach : the tangram approach}

\cite{wagenaar_using_2016}
CITE Higgs (matter questions leading to methods improvement)

%% Lecuyer 77 : Villermé et autres praticiens pratiquent en contemporains des grands théoriciens que sont Laplace, Fourier et Poisson. Mais sont plus praticiens. Mettent en pratique quelques outils mais surtout développent un savoir faire.





Using this framework appears interesting to understand interventions for public health. Expliquer Solthis.

a. How does it contribute to evidence generation
b. How does it contribute to the development and emergence of statisticians as a profession
c. How does it contribute to the scientific and administrative independence of the country
d. How does it contribute to new methods development


1. Entry through EMR. Not sufficient.
2. M\&E approach
3. let's do Tangram

HIS are complex political and technical objects. We contend that the creation and implementation of functionning systems should be through combination of these different mindsets. Most importantly, we argue that building health information systems should be the result of complex national evolutions. As much as African philosophy is claiming a space in occidental accademia and curricula, there should not be a tendency to apply one size fits all solutions to complex and varied situations, and there is a space for local innovation, invention. We offer this reflexion as a guide for project manager and field workers to guide their reflections and work in the long term of building healht information systems and not only applying standardized methods.

% Lecuyer 77 : devt methodo de statsitique sanitaire en france vient beaucoup de milieu adiministratif. Direction particulière de recherche par conséquent. =>> Need to apply research appraoch in administration. Not impose and deliver models.
% Lecuyer 77 : Villermé fait recours à de multiples données pour avoir ses conclusions


%\addtolength{\textheight}{-12cm}   % This command serves to balance the column lengths
% on the last page of the document manually. It shortens
% the textheight of the last page by a suitable amount.
% This command does not take effect until the next page
% so it should come on the page before the last. Make
% sure that you do not shorten the textheight too much.

%%%%%%%%%%%%%%%%%%%%%%%%%%%%%%%%%%%%%%%%%%%%%%%%%%%%%%%%%%%%%%%%%%%%%%%%%%%%%%%%

% \section*{ACKNOWLEDGMENT}

%%%%%%%%%%%%%%%%%%%%%%%%%%%%%%%%%%%%%%%%%%%%%%%%%%%%%%%%%%%%%%%%%%%%%%%%%%%%%%%%

%\newpage
\bibliography{bibliographie}




\end{document}
