\section{Conceptual framework}

The need for to produce and publish high quality information is now recognized as inevitable for most modern organizations. Authors have recognized the early role of statistics in the invention and installation of democratic governance \cite{porter_trust_1996}, and their importance for the management and strategic governance of most organizations. The faith in the capacity of the use of numerical data to empower individuals and organization to reach their full potential in the developing work has even led some to declare the advent of a \textit{Data Revolution}\cite{independent_expert_group_on_a_data_revolution_for_sustainable_development_world_2014} \cite{center_for_global_development_delivering_2014}.

In the Global Health field, the importance of Health Information Systems (HIS) has been claimed, and these systems have been deemed essential to proper decision making and administration of health programs \cite{abou-zahr_health_2005}. Meanwhile, the weakness of health information systems, and the inadequacy of current systems in a lot of  developing countries is widely recognized, and has fueled the development of a dedicated academic literature genre \cite{abou-zahr_better_2010} \cite{kiberu_strengthening_2014}. Meanwhile, the health sector shouldn't be too embarrassed on the state of its information generation capabilities, as other fascinating bodies of works describe the weakness of information provided in other sectors of public life as well \cite{jerven_poor_2013}.

Meanwhile, this literature, if it questions the performance of HISs, does not question the notion of their possibility. The existence anywhere in a the world of a coherent, purposive and purposively designed systems producing all information relevant for the strategic planning and daily management of health systems is nonetheless far from evident. It is often unclear even for practitioners, what HIS are or are not. As systems, health information systems are indeed recognized as a complex ensemble of tools, methods and processes aimed at answering a diversity of information needs for actors in charge of a multitude of decisions in health systems. Meanwhile, defining __what__ should be happening in HIS and __how__ it should happen is not as straightforward as it may appear.


%% Add references


%Morten Jerven challenge of the way fundamental economic indicators are measured is a call for caution in the use of widely used numerical figures as well as it is a radical critic of the systems in place to produce these figures. Jerven shows how the history and politics of developing countries have an impact on how statistics are made. This is especially true in the health systems in developing countries, where both the objects being measured and the structures in place for this measure are the products of complex and intertwined social and political histories.

%% Why paper interesting.

%% A REVOIR---------------------

In the effort to provide guidance and advice on how to produce much needed information for health systems, Health Information Systems are presented as unified systems, with inner logics and methods. Frameworks are provided, that present guidance and advice on how to design and administer HIS \cite{health_metrics_network_framework_2008}. These documents are very helpful for practitioners and field workers, but there is an inherent risk in these normative, which is to provide an illusion of unity, coherence and inevitability in some systems. The definition of requirements both in terms of purposes and tools of HIS results lowers the incentives to problematise this information and to limit the role of HIS strengthening to answering these requirement more than using local and specific assets and problems to find local solutions. In other words, once a standard is defined, all questions asked only appear to be technical questions on how to achieve and enforce this standard, to the detriment of problematization and customization of these standards.%% \cite{bergeron_savoirs_2014}  le savoir et l'expertise sont "un réseau qui met en relation toutes sortes d'acteurs, de dispositifs et d'instruments, de concepts et d'arrangements institutionnels et spatiaux"


There is a false unity in the description of Health Information Systems [HMN , Vital Wave , AEDES]. The production and use of information a population, its health and the health systems it has access to indeed has a long and varied history, with different traditions and intellectual origins. The concentration on short term interventions or projects, aimed at strengthening specific aspects of these systems fosters a loss of perspective when it comes to understanding the globality of a system. This is especially true in countries with limited national governemental resource and capabilities, where it is even harder to enforce a unified approach to producing this information. In many settings, the repeated design of strategic information plans and other M&E plans is happens in the meantime as the implementation of atomised and unstructured subsystems, and the piecemeal collection of  evidence and data by a multitude of actors.

Understanding the differences in premices and historical precedence of different approaches appears essential in empowering actors to understand and objectify choices, shortcuts and traditions, and to imagine innovative approach.

There are indeed multiple mindsets and traditions when it comes to the quantification and quantification of public health. These are generated from the history and traditions of statistics both as a scientific and an administrative field. The field of public health has also generated different traditions and approach. In former colonies, these traditions and mindsets have been compounded by the colonial experience, which created its own set of values and practices when it came to population statistics and the use of evidence for administration. Finally, the emergence of new technologies and the generalisation of powerful computing methods have deeply modified the way statistics are thought and made, and this in turn put a final layer of complication in the way health information is considered in the developing world.

This paper will offer a quick perspective into long term trends that ended up producing contemporaneous HIS questions. We will do so building on a litterature close to the Science Studies movement and Colonial Studies. We feel this body of research is especially important in the field of Global Health, in which the time frame is often the time frame of the project or the planification round, and where questions are rediscovered and solutions reinvented with often an historic short-sightedness, with little perspective or critical insigth in previous practices.

From this review, we will offer an approach to questioning HIS related intervention, and to understand how they can contribute to fostering evidence generation and use in developing countries.

\cite{vital_wave_consulting_health_2009}. Collection of methods and traditions from different fields to create multi-morphous knowledge. %\cite{bergeron_savoirs_2014}




%%---------------------------------

% Statistiques sont directement élaborées dans le dessein de serivr le politique. Contrairement à d'autres formes de savoir (Hacking)



\subsection{Health Information Systems as public statistic systems}

"reliable and timely health information is the foundation of public health action" \cite{health_metrics_network_framework_2008}.

The use of information for public decision and policy making is not specific to health systems, and methods and approaches to statistics have been developped concurrently in the health sector and for the purposes of economic administration, education and crime [Histoire de la statistique]. Statistics = Staat. Meanwhile, the mobilisation and usage of information is part of the genesis of public health, from the tutelar figures of John Snow, Florence Nightingale and Carr, to the importance of statistics in the French hygienists movement	\cite{porter_trust_1996}, and more specifically the figure of Villermé who 'creates the fusion between hygiene, statistics and the study of social world and its evolutions'.

In France, From its initial development, the \textit{Annales d'hygiène} contains articles that touch to the diversity of interests of public health : general demography, epidemiological statistics, study of healthcare institutions, and study of different social and environmental problematics (LECUYER). Compare to HMN ?

Meanwhile a prescriptive technical approach should not miss the fact that the most mature HIS are the result of technical and social tools, that have evolved over decades. One should note that the exploration of these realities were mainly driven by interest of scientists of public deciders willing to tackle the historic challenge of urban poverty and insalubrity, methods evolved during these very stages of study (Nightingale and mortality).

Not only did the methods get developped during this time. Alain Desosières showed how the development of methods happened at the same time as statistics developedThe structuration of statistics as a profession, and how different natioanl statistics cultures were developed according to  Different traditions.

List different methods that appeared.

%Lecuyer : statisticien and hygienists are a goid carrier choice for the medical profession => get good scientists

It is thus important to understand how the development of methods and practices is linked to specific political and institutionnal situations.

%% jusqu'à WWII health policy is primarily influenced by pasteurian paradigm. After WWII, renewal of epidemiliology. France = more of a mathematical and modelization academic discipline. In the US, more oriented towards action. Note : admin & scientist ? \cite{bergeron_savoirs_2014}

Foucaldian perspective

\subsection{The heritage of colonial statistics}

In sub-Saharan Africa, the way health information systems have evolved is heavily influenced by both medical traditions of colonizing powers, and their statistical traditions. Meanwhile, the development of statistics in these countries was also contrived by the colonial phenomenon as a whole. Comparing the evolution of cartographic knowledge in the French colonies, we can see how the logics that defined mapping \cite{appadurai_number_1996}.

In Europe, influence of social statistics for welfare design. Minor role in colonial states (Cite Cordell Ittmann  Maddox)

Colonial statistics was using rudimentary estimates, compiled by less trained administrators and outside observers (Cite Cordell Ittmann  Maddox) with unclear incentives (Gervais and Mandé). Only in the 30's are methods converging with the ones used in the metropole.

Space defintiion and toponimy influenced by colonial classification and choices (Gervais & Mandé + article histo carto). This set the framework for later denumeration by setting cuonting units. Lasting impact.

And then Global Health. African populations as object of external description, Cite Bonneuil

\subsection{Health Information Systems in the age of computerized data collection}
%OpenHIE

Development of EMR

ABSENT = outillage Statistique. = > Le tournant de Global Health Metrics ? \cite{health_metrics_network_framework_2008} promotes global frameworks and norms.
Paper vs computer debate

Sampling and then big data. New methods.


\susbection{A typology of HIS interventions}

Understanding the long term trends in thinking and producing information on the health of population and health systems is important for organisations and people currently working health information systems strengthening projects, in order for them to understand their positioning and implications. Long term phenomenons are indeed still present in the way we think about how to improve evidence generation in health systems. There are multiple ways in which health infroamtion systems intervnetions in developing countries are being made. We hereby offer a simple typology of these interventions, and trace each intervention type to corresponding frameworks.

Taking in consideration a variety of Health Information Systems intervention in developing countries, we differentiate three main approaches in these programs. In this exercise as in a lot of situation, there is of course no intervention that can be considered pure type, and any project will be using forms and practices from different approaches. Meanwhile, understanding when and why different solutions are geared from different mindset appears essential for understanding the promises and shortcomings of different approaches.

\subsubsection{Process oriented Approach : the puzzle approach}

A first type of interventions is putting emphasis on the systemic dimension of HIS, and focuses on strengthening processes and organisation of HIS. This is a mainly goals oriented approach, in the sense that the starting point of these interventions is usually the defeinition of information needs for administrators or end users of this information.

We call this appraoch the jigsaw-puzzle approach, because of the way it provides information. From an overarching image that one wants to reconcile, pieces have been cut and have on and only way to be assembled to provide the picture we want to see. In this sense, the jigsaw puzzle is very much a social and organized experience (in the words of Georges Perec : "puzzling is not a solitary game: every move the puzzler makes, the puzzle-maker has made before")

This approach will put an emphasis on the use and institutionnal usage of information. A weakness of such interventions is their rigidity, and their heavy determinism. Producing normative docuemtns, frameworkds and guidelines and gearing for standardization.

Multiple projects can use this approach. RHINO, HMN, M&E. Some patients files design.

Does not make

\cite{rhino_introducing_2003}

\subsubsection{Data Collection Approach : the pixel approach}

A second type of projects and interventions is putting great emphasis on the collection of data. These approaches are usually relying on more technical perspectives. ODK , OpenMRS,

This approach is built on the belief there is a fundamental value of data collection, and more data is better, and information systems are first and foremost data collection tools that have to be optimized and should perform. These approach can be fostered by programs that value first and foremost individual patient care,

Usually computer based, but not always. Risk is the undervaluation of politidal determinants.

\subsubsection{Computing Approach : the tangram approach}

\cite{wagenaar_using_2016}
CITE Higgs (matter questions leading to methods improvement)

%% Lecuyer 77 : Villermé et autres praticiens pratiquent en contemporains des grands théoriciens que sont Laplace, Fourier et Poisson. Mais sont plus praticiens. Mettent en pratique quelques outils mais surtout développent un savoir faire.





Using this framework appears interesting to understand interventions for public health. Expliquer Solthis.

a. How does it contribute to evidence generation
b. How does it contribute to the development and emergence of statisticians as a profession
c. How does it contribute to the scientific and administrative independence of the country
d. How does it contribute to new methods development


1. Entry through EMR. Not sufficient.
2. M&E approach
3. let's do Tangram

HIS are complex political and technical objects. We contend that the creation and implementation of functionning systems should be through combination of these different mindsets. Most importantly, we argue that building health information systems should be the result of complex national evolutions. As much as African philosophy is claiming a space in occidental accademia and curricula, there should not be a tendency to apply one size fits all solutions to complex and varied situations, and there is a space for local innovation, invention. We offer this reflexion as a guide for project manager and field workers to guide their reflections and work in the long term of building healht information systems and not only applying standardized methods.

% Lecuyer 77 : devt methodo de statsitique sanitaire en france vient beaucoup de milieu adiministratif. Direction particulière de recherche par conséquent. =>> Need to apply research appraoch in administration. Not impose and deliver models.
% Lecuyer 77 : Villermé fait recours à de multiples données pour avoir ses conclusions


% _______________________________________________________________________________



A first approach to HMIS is a consideration of the stated goals of these information systems. Figure \ref{HMISGoals} shows what these main goals are.
\begin{figure}[htp]
	\centering
	\begin{tikzpicture}[node distance=2cm]
		\coordinate (A) at (-4.5,0) {};
		\coordinate (B) at ( 4.5,0) {};
		\coordinate (C) at ( 0,7.7942) {};
		\draw[name path=AC] (A) -- (C);
		\draw[name path=BC] (B) -- (C);
		\draw (1.1,5.8971)--(3.5,5.8971) ;
		\draw(3.5,5.8971)--(3.5,3) ;
		\draw [->](3.5,3)--(2.77,3) ;
		\node at (4.4,4.25) {Feedback};
		\foreach \y/\A/\txtHigh in {0/Patients Care/0.8 ,2/Facility Administration \\ and Reporting/2.5,4/Planning \\ Monitoring \\ \& Evaluation /4.8}{
			\path[name path=horiz] (A|-0,\y) -- (B|-0,\y);
			\draw[name intersections={of=AC and horiz,by=P},
			name intersections={of=BC and horiz,by=Q}] (P) -- (Q)
			node[align = center,above] at (0,\txtHigh){\A};
		}
	\end{tikzpicture}
	\caption{Information needs for HMIS}
	\label{HMISGoals}
\end{figure}


\begin{description}
	\item[Patients Care] Taking care of patients is the primary goal of a health facility. To do so, it is necessary to collect data on these patient, data that will be transmitted (to other services), stored and reused during further follow-ups.
	%BIBLIO chaperon88 medical most important et doit diriger le reste
	\item[Facility Administration and Reporting] At facility level, HMIS data is used in daily activities to quantify and forecast needs in health inputs, and to create reports for higher levels of the health system.
	\item[Planning, Monitoring \& Evaluation] People in charge of the administration of health systems at local or national also need data to monitor activities in the health system, to evaluate the results of interventions, to report to funders or to plan later interventions.
\end{description}

The different needs for information can roughly be thought of as being the needs of different type of actors of the health system. Meanwhile, this understanding is not fully true, as at local level, actors will often hold multiple roles and will thus have to use information in different situations. For example, a physician may also be in charge of managing his health facility, and will thus need to plan activities and report on them.



\subsubsection{Functional approach}
\label{sec_function}

A first way to approach HMIS is to describe the four principal functions that are necessary to have a HMIS to run. Figure \ref{HMISFunctions} presents a simplified sketch of the principal functions that are to be filled in order for HMIS to produce useful information.

\begin{figure}[h]
	\begin{center}
		\begin{tikzpicture}[node distance=.8cm,  start chain=going below,]
			\node[punktchain, join] (DataCollection) {Data Collection};
			\node[punktchain, join] (DataManagement) {Data Transmission and Management};
			\node[punktchain, join] (DataAnalysis)   {Data Analysis};
			\node[punktchain, join] (DataUse) 		  {Data Use};
		\end{tikzpicture}
	\end{center}
	\caption{Different functions inside the Health Information Systems}
	\label{HMISFunctions}
\end{figure}

\begin{description}
	\item[Data Collection] Primary data collection is essential to the production of any information system. In the case of HMIS, data collection happens in health facilities, and is made by health professionals. Data collected in health facilities can be individual patient data collected in patients files or cards. It can also be a first level of aggregation of this data, as for indicators that are reported on a regular basis by facilities to higher levels of the health system. This reporting usually happens through standardized reports, that are then transmitted by successive aggregation to the top of the health pyramid.

	\item[Data Management] Data collected in health facilities has to be stored and archived, to be later accessed and reused. Data management work can encompass managing paper data, or managing computerized data. Individual patient data will be computerized in \gls*{emr} whereas aggregated indicators are stored in data-warehouses, like the DHIS2 software.

	\item[Data Analysis] Data that is collected and stored in HMIS can then be analyzed. Analysis can be defined as the transformation of data into information. The results of data analysis can be varied, from collection of graphs and maps that can constitute dashboards, to the results of complex models that provide evidences of causality.

	\item[Data Usage] Information is the end product of the HMIS, and is used by decision makers or health workers to achieve their tasks. For example, a nurse in a health post may need the monthly consumption of a health product to place an order. A District Health Officer may consider the evolution of monthly number of cases of malaria in his district to plan malaria prevention activities. At national level, a worker at the Ministry of Health may use the number of patients tested positive for HIV to design grant applications for the Global Fund.
\end{description}

Even though the schematic representation of this functions is linear, it should be noted that this linearity is not true in practice. Even once data has been analyzed the results of this analysis has to be archived, and transmitted to the information end users. In some situations, data can be used in its raw form. For example, a physician may use a biological result that he has received in a raw form. Finally, for some data, data collection won't happen inside the health  facility. For example, population survey results may be used to plan and target some interventions, but primary data collection will have happened outside of the health system, and the first function to be used in the HMIS will be the data management function.

The way a program considers and plans each of these functions will define how a specific HMIS will work. We will now present three common approaches to HMIS.


\subsection{A typology of HIS approaches}

Based on this historical, we tend to describe different approaches to HMIS in developing countries, and different approaches and to present a typology of strengthening strategies.

\subsubsection{Three HMIS archetypes}

%Functions of HMIS (cf. section \ref{sec_function}) are not independent of each other. Defining the relative importance of different functions of HMIS in the overall systems can change greatly the way a HMIS functions, and the output it produces. We differentiate three paradigmatic types of HMIS, varying on the respective influence of different functions. Building on the idea that a HMIS is used to provide an image of the activities and performances of a health system, we describe each function as a different way of making an image.

%\paragraph{Jigsaw Puzzle HMIS} - A common way to design HMIS in developing countries can be considered as a Jigsaw Puzzle approach. A series of indicators are designed by program managers. These indicators are deemed to be \textit{sensitive} and \textit{specific}, and are supposed to allow managers to track and identify precisely the performance of health systems, and to provide important information on health system's results. The HMIS will then be organized to produce carefully designed indicators at facility level, and to transmit these indicators to higher levels for aggregation.

%In these types of system, a lot of importance given to data collection functions, as the quality of this primary data collection is key to the rest of the work in the system. Data management in these system is often limited to aggregating some data and transmitting it to different actors in the health systems. Data analysis is usually mainly descriptive and is limited to presentation of time series values or mapping of indicators along administrative boundaries.

%These systems are similar to jigsaw puzzles, made of specific pieces, to compose a predetermined picture. When they are well designed, these systems can provide useful information on health systems. Meanwhile, they are vulnerable to any variation in primary data collection. As for jigsaw puzzles having a piece missing will jeopardize the possibility to get the whole picture right.

%\paragraph{Pixel HMIS} - Another way to conceive HMIS is built on the collection and use of a multitude of individual data collected through  \gls*{emr}. Once the data is collected, program managers can query different indicators on different levels of aggregation, that can be extracted from different EMRs. In the best situations, interoperability of multiple  \gls*{emr} present in a country allow for a central analysis of the data \cite{pugliese2009large}.

%These systems allow a great variety of analysis, with a great variety of approaches. Analysis can be led varying geographic and time focus, or changing definitions of computed quantities. It also allows longitudinal analysis that are more difficult to perform with other approaches.

%This approach thus involves a great investment in primary data collection and management, and allows elaborate data analysis. Meanwhile, it requires a technological investment and maturity that is seldom achieved in rich countries, and thus is very rare in developing countries.

%\paragraph{Tangram HMIS} - Between the two extremes that are puzzle and pixel HMIS is a third, less prevalent approach to HMIS. This approach will be compared to the tangram game, in which simple forms are used and reused to draw different pictures. In this approach, the emphasis is put on the management functions of HMIS. Simple data elements are collected and stored, and are used and combined in different ways depending on the analysis that is done.

%A key component of this model is thus the ability to store and reuse data, thus putting an emphasis on the middle tiers of data management. The use of data warehouses for computerized data is thus a characteristic of this approach. Meanwhile, it also requires an emphasis on data analysis in order to provide relevant information to end users.




\subsection{HMIS strengthening strategies}

Depending on the HMIS model that is used, programs will implement different type of HMIS strengthening approaches. Programs who privilegy a jigsaw puzzle approach to HMIS will tend to focus on standardizing procedures and methods for data collection and data analysis. Meanwhile, programs who privilegy a pixel approach to HMIS will tend to favor solutions geared towards the implementation of new and performing data collection tools.

\begin{description}
	\item[The institutional approach] operates under the assumption that all functions of information systems should be geared towards and submitted to the end information users. This approach tends to be extremely normative as any activity in the information system has to be oriented towards one main predefined goal. In doing so, this approach undervalues the benefits of both the integration of external data, and the positive externalities data collection and analysis may have on multiple users.

	% BIBLIO thieren05 besoin d'harmonisation #InstitutionalApproach
	% BIBLIO chaperon88 1958 reform of hospital morbidity collection forms from methods at Hotel Dieu : elle va tres rapidement echouert: le questionnaire demande un sercroît de travail aux services administratifs confrontés dans le même temps à une nomrlasation de l'ensemble des imprimiés : côté médical l'investigation est ressentie comme un contrôle de l'administration sur l'activité médicale. (FROM RAZPPORT ECOLE DES MINES)
	% Tradition française : médical prime...

	% BIBLIO braa07 "the top-down and all-inclusive approach to standardization common among ministries and central agencies"
	% BIBLIO braa07 "the individual standards must be crafted in a manner which allows the whole complex system of standards to be adaptive to the local context"
	% BIBLIO braa07 principle of flexible standards, and the principle of integrated independence.
	% BIBLIO braa07 we can get rich information from minimal data. A focus on the must-know rather than nice-to-know information, as illustrated by the emphasis on indicators and the minimal essential data set, can be extremely powerful in that it allows a simple, well-chosen data element to be used for several purposes.
	% BIBLIO braa07 Second, accept that there will always be technically incompatible subsystems.

	\item[The technological approach] relies on the assumption that collecting data and making it available is a sufficient enabler for all other functions of information systems to operate. In this sense, a direct link is made between an information need and a data gap. This approach comes at a cost, and provides only limited benefits if it is not supported by improved data analysis. These solutions tend to provide highly specialized and siloed data collection systems.
	% CITE macfar05 "The availability of powerful computers and of reasonably priced software has led to a proliferation of database systems" #ICTBased

\end{description}

We argue these two approaches focus on the most expensive ways to strengthening HMIS (data collection and systemic reforms), and are emphasizing the design of systems and tools that are specific to precisely defined data needs, thus limiting the possibility to implement secondary data usage and the positive externalities of their interventions. The archetype of these pitfalls are the well known parallel and siloed data systems present in many developing countries health systems.

Meanwhile, some of the most significant successes in the strengthening of health information systems in developing countries have been reached precisely through the strengthening of their middle tier. The District Health Information System (DHIS2) project has become a pervasive system to store and organize data collected in developing countries health systems. The DHIS2 approach to health information is based on the organization and storage of multiple data types and sources in a generic data warehousing approach. Its versatility and its ability to adapt to different contexts and data has made it increasingly used in multiple context, thus arguably improving the storage and the availability of health data in low resources countries. Other approaches geared towards the promotion of interoperability of different dimensions of data systems, such as the Open Health Information Exchange framework are also gaining traction.

% CITE OHIE

If these approach have provided efficient solutions to organize and access HMIS data, there is still a lack of solutions to analyze and use HMIS data. Indeed, the high dimensionality of HMIS data and its average low quality make it essentially hard to analyze using standard methods available in developing countries health systems. We will now describe the research project developed to explore ways to analyze this data.

We want to ancher hospital data in a multiplicity of data sources that are available to health professionals to create evidence and make decisions. This data can be metadata that is routinely generated during data collection exercises. It can also be data generated by other organisations of the community. Finally, it can be similar data generated by different actors.

\subsection{Approach and research questions}

% BIBLIO walsh07 Structuration des questions IS in dvt countries in 4 dimensions : #ICTBased
%% 1. Link ICT and Development
%% 2. Cross cultural aspect of ICT
%% 3. Local Adaptation
%% 4. Marginalized groups

This project aims at developing new tools to use heterogeneous data sources to provide information for health systems deciders. We will thus explore methods for \textit{data hybridization}

We will look into different types of data hybridization :
\begin{\begin{description}
		\item[System generated data] The computerization of data is seen as a benefit in terms of efficiency and quality of collected data. An undervalued benefit of computerization of administrative data collection

		\item[Interoperable data sources] One of the banes of HMIS in sub-Saharan Africa is the multiplicity of parallel data systems, resulting from a multiplicity of vertical health programs in those countries, generating parallel data fluxes%BIBLIO HMN.
		There is a lot of work and talk currently underway to implement interoperability between different data systems. The questions surrounding interoperability are usually ex-ante questions of standards for the design of unified systems. Meanwhile, the ex-post question of the analytic benefit of having interoperable systems is seldom worked upon, even though this is the question that will define the conditions of adoption of the said standards.

		\item[External data sources] Using data sources non traditionally included in the HMIS framework to generate health related information is currently an important field. At a macro level, the Global Burden of Disease is a case of extreme hybridization of XXX data sources of different types. % BIBLIO GBD
		Other projects use data sources of very different types, like UUU and UUU for malaria prevalence, usage of cell-phone CRV for malaria diffusion estimation, or population distribution. % BIBLIO Catherine / Tatem
		Meanwhile, there are little examples of this external data usage being used for decision making by local professionals.
		\end{description}}





	The generic question we ask is : How can data currently routinely used data sources be used to provide actionable information for decision makers ?

	More precisely, we will ask three main data analysis questions :
	\begin{itemize}
		\item How can metadata collected in an EMR be used in EMR data analysis ?
		\item How can different non standardized HMIS sources be mapped and jointly analyzed ?
		\item How can multiple data source be integrated to HMIS and analyzed to provide information at local level ?
		\item How do decision makers in health systems consider HMIS generated data, and how does it influence the way HMIS are engineered ?
	\end{itemize}

	% BIBLIO macfar05 : Type d'organisation HMIS depend de culture locale et type d'organisation statistique nationale
	% BIBLIO macfar05 : Difficultés pour faire système statistique local :
	%% 1. diversité des sources
	%% 2. Local data needs vs central
	%% 3. Harmonization vs disaggregation
	%% 4. Health Resource
	%% 5. Confidentiality
	%% 6. Incentives
	% BIBLIO macfar05  Technology "The Availability of powerful computers and of reasonably priced software has led to a proliferation of database systems"  #SolutionType

	%BIBLIO chaperon88 la nécessité d'une conception souple, non exhaustive et évolutive de tels systèmes privilégiant les indicateurs multiples adaptés aux différents niveaux de responabilité et aux différents objectifs #NormeFaible

	list projects

	Categorize through type of data and methods used.

	%TODO vision unifiée sur outillage statistique => Approche bayesienne
	%TODO on considere que progres dans HMIS ne peut pas venir d'un progres dans une seule dimension. On fait donc un projet sur tous les tableaux en meme temps

	%%%%%%%%%%%%%%%%%%%%%%%%%%%%%%%%%%%%%%%%%%%%%%%%%%%%
	Our method is taken in a decision theoretic framework. We do not aim at providing substantial knowledge with the results of our analysis, but rather we aim at providing information that will emopower decision makers to take decisions.

	Cadre bayésien %%BIBLIO SPIEG04 Decision theoretic framework.
	%% CITE SPIEG04 "Since advances in health-care typically happen through incremental gains in knowledge rather paradigm-shifting breakthroughs, this domain appears particularly amenable to a Bayesian perspective."

	We operate in decision theoretic framework, our end objective being to inform decision making for the

	Approche Probabilistique des données

	Control Charts
	%%%%%%%%%%%%%%%%%%%%%%%%%%%%%%%%%%%%%%%%%%%%%%%%%%%%



	Each of these questions explores a different aspect of how standard HMIS data analysis can be expanded to produce useful information with HMIS data. Metadata like the times of creation and savings of EMR forms are indeed seldom used. Meanwhile they provide useful information on how data is collected, and on the working patterns inside health facilities. Our second question explore a different problem, which is often taken as a question of interoperability. Indeed the multiplicity of programs and actors working in many health systems generates a multiplicity of indicators used, that are often related but not identical. There is a need for simple and effective methods to map and conjointly analyze data from different HMIS systems. Finally, HMIS should be useful for local level analysis and decision making. Meanwhile, HMIS data is seldom useful on its own and has to be integrated in larger analysis frameworks to produce interesting information. This integrating can be difficult at national level, but it is even more complicated at local levels, has mapping precisely different data becomes more and more of a challenge at small scale.

	We also aim at providing a critical evaluation of the way HMIS are thought of in developing countries societies. Authors like Alain Desrosières and Ted Porter have shown how statistics and computation have come from and generated different cultures of public action in modern societies. The ambivalence of numbers as descriptors or norms has an influence on how information systems are thought of as top-down normative systems instead of knowledge systems. We aim at interrogating how this perception has its roots in long term local historical trends as well as in the tradition and methods of quantitative public health.


	To answer these question, we will conduct four distinct research projects.
	\begin{description}
		\item[Aim 1] Evaluation of the benefits of improved data collection for HIV patient care in Kenya.
		\item[Aim 2] Test of multiple semantic approaches to interoperability in Bénin.
		\item[Aim 3] Definition and test of a local malaria elimination metric in Namibia.
		\item[Aim 4] Analyze of the theory and practices of Health Information Systems for national decision makers in Mali.
	\end{description}

	These aims have been designed to provide insights on the problematic posed for each HMIS goal and function. We will now describe each of these aims in more details.
