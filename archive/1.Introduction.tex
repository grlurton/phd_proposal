\section{Introduction}


%% BIBLIO Shibuya05 "Health Statistics (...) are the basis for every aspects of health planning"
%% BIBLIO Shibuya05 Accountability stemming from MDGs
%% BIBLIO Shibuya05 WHO Dependent on data from countries that is usually bad quality #DataQuality
%% BIBLIO Shibuya05 Since 2003 WHO makes effort for better primary data collection #DataQuality
%% BIBLIO Nolen05 information essential to identify and understand health inequities #WhyHMIS


%% BIBLIO Mutemwa2006 Local level needs for "effective strategic Decision" #WhyHMIS
%% BIBLIO Mutemwa2006 "The argument for HMIS are not based on unequivocal empirical evidence, or tested theory, that the information carried in HMIS makes a difference, but rather represents a normative view of management capacity"
%% BIBLIO Mutemwa2006 Other types of evidence are mobilized during decions makin "observational, training, experiental" , "gut feeling, hearsay and ad hocry" + no really clear pattern on how HMIS info is used for different types of decisions / different steps
%% BIBLIO Mutemwa2006 Decision making is Problem Recognition, Investigation and Solution Development

%% BIBLIO Boerma2013 HIS is essential for district health teams to effectively plan and managage Health services  #WhyHMIS
%% BIBLIO Boerma2013 fonction de HIS : reporting, planning and M&E
%% BIBLIO Boerma2013 HMIS = juste from health facilities
%% BIBLIO Boerma2013 Pb = rationalization and RH #DataQuality
%% BIBLIO Boerma2013 solution technique (IT) & orga (RH / rationalization of data collection)

%% TODO Different questions to explore to help ancher HMIS information in decision making : How does HMIS directly contribute to health of patients ? What is the degree of confidence in the data we have, How to sum / summarize every bit of information we have, What additional information will we need .... How do we convey all this information.

%% TODO Hospital data collection comes from a mindset that is foremost frequentist and administrative (DESROSIERES). With emergence of global health and global solutions (IT), comes to a form of system thinking that creates expost theory around SIS (Lippeveld, HMN). Latest iteration is more tool based. lacks a measurement framework. Not facts based. In developped countries, success are hard and linked to administrative and local specific questions. Not same solutions in developping countries. There is emergence of a new model based approach to measurement for global health, recognizing fundamental incompleteness / problem fo available data. Meanwhile, M&E essentially stays in simple mechanistic frequentist framework. We will work on exploring a new approach to measurement in HMIS.

If a literary form had to be chosen to write or talk about \gls*{his}, the complaint would probably be everyone's favorite pick. Be it complaints on the burden of work involved in collecting, managing and analyzing data in health systems, or laments on the inexistence of good quality data in most developing countries health systems, HIS are usually described as a non performing burden of health systems, that can only be improved [HMN citation]. This frustration has multiple causes, and is only matched by the expectations placed in HIS and their widely recognized importance, some authors calling HIS "the foundation of public health"\cite{abou-zahr_health_2005} . Collecting and analyzing information on activities and results of health systems and on the populations served is indeed essential to guide strategic decision making and to inform health policies.

National HIS are complex, in the sense that they are expected to provide a wide variety of information, acquired from a wide variety of data sources. Producing this information requires the contribution of a multitude of actors, and is a huge organizational and methodological challenge. Data used to produce relevant health information may come from administrative records, organizational documents or population surveys, and are produced by a variety of actors and organizations, with differing cultures and approaches.
% TODO lister les différents types d'info à produire.
% BIBLIO HMN Network
% BIBLIO walsh07 in IS in PVD issue, 3 / 4 papers are on health

Moreover, HIS have to be able to adapt rapidly to changing epidemiological, organizational or political situations. They have to be able to produce relevant information on emerging health issues, and to adapt to the entry of new actors or to a modification in the mode of management of health systems, or to new priorities or questions.


%% TODO editer ce paragrpahe. faux.
In richer countries, the issue surrounding health information is often one of regulation and standardization. The  existence of well performing and well established data sources on populations, and the relative ease of collecting massive amounts of data on individuals pose questions that are mainly related to the protection of privacy, and to the definition of standards for interoperability. The definition of what information should or can be produced is usually a legislative and matter, handled by dedicated entities.

%% BIBLIO Nolen05 Need to identify and understand inequities make it necessary to collect info on "specific health measures and equity stratifiers". #MoreData
%% BIBLIO Nolen05 Need to increase volume / diversify data collected. #MoreData #DataQuality

In developing countries, where population data is scarce and data collection can be much more of a challenge, the issues are much different. Health policies depend heavily on the financial, political and technical support of international actors, with differing orientations and priorities. As a consequence, developing countries \gls*{his} strengthening programs can be classified into two great families : technology based solutions and institutional reforms. The first family is mainly targeted towards improvement of data collection and relies on solutions relying on the revolution in  data collection capabilities. The second family relies on the adequation of health information produces to policy makers needs, and usually promotes a top-down and normative approach to Information Systems.


If each of these approaches has its benefits and some success stories,  they appear to miss an important part of what makes information systems work. They put their emphasis to the extremes of the information production chains (cf. Figure \ref{HMISFunctions}) and undervalue the middle tiers of these systems. This proposal designs research program oriented towards exploring ways in which this middle tier can be mobilized to improve the value of information produced by HMIS.

This document will start by an introduction of the conceptual framework surrounding the proposed research. We will define Health Management Information Systems, using two classic schematic approaches. We will then define the objectives we will pursue in our research, and will finally describe our different aims and the methods we want to use to complete them.




%% TODO \footnote{There should be thinking about what constitutes the coherence or the platform of a Health Information System. It could sometimes be considered the nature of information stored in HMIS (regarding health) makes this coherence. However, as << idee est que on a }
